\section{Solving Identification with a Control Function}
\label{sec:controlfun}

If you could control for $U_i$, then you would.
Laffers et al, for example, tests sequential ignoability.




The IV literature assumes a first-stage monotonicity condition, where randomised $Z_i$ influences mediator $D_i$ in at most one direction.
\begin{definition}
    \label{dfn:firststage-monotonicity}
    First-stage Monotonicity \citep{imbens1994identification}.
    \begin{equation}
        \label{eqn:firststage-monotonicity}
        \Prob{ D_i(1) \geq D_i(0)} = 1    
    \end{equation}
\end{definition}

Assuming \ref{dfn:firststage-monotonicity}\eqref{eqn:firststage-monotonicity}
in a mediation setting opens mediation to the wide literature on IV and selection models for identification in the presence of selection. 
\begin{theorem}
    \label{thm:selection-model}
    Under monotonicity, mediator $D_i$ can be represented by a selection model. \\
    Suppose \ref{dfn:firststage-monotonicity}\eqref{eqn:firststage-monotonicity} holds, then there is a function $\mu(.)$ and random variable $U_i$ such that $D_i$ takes the following form.
    \[ D_i(z) = \indicator{ \mu(z) \geq U_i}, \;\; \forall z = 0,1 \]
\end{theorem}
\begin{proof}
    Special case of the \cite{vytlacil2002independence} equivalence result; see \autoref{appendix:selection-model}.
\end{proof}

\autoref{thm:selection-model} is a powerful result: it says that at the cost of assuming monotonicity (as is done in the IV literature), then selection into $D_i$ takes a latent index form, and opens up identification in a mediation context to the wide literature on identifying treatment effects in selection models.
