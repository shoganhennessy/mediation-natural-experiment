\section{Solving Identification with a Control Function}
\label{sec:controlfun}
If your goal is to estimate CM effects, and you could control for unobserved selection into $D_i$, then you would.
% Alas, $U_i$ is by definition unobserved.
The control function method takes this insight seriously, and provides minimal conditions to model and control for the implied unobserved confounder, $U_i$.\footnote{
    This section does not improve on the control function approach, instead only noting its utility to solve the identification problem of CM in a natural experiment setting.
}

For notation purposes, suppose the vector of control variables $\vec X_i$ has at least two entries.
Denote $\vec X_i^+$ as one entry in the vector, and $\vec X_i^-$ as the remaining rows.

\begin{definition}
    \label{dfn:controlfun-assumptions}
    Control function assumptions.
    \begin{align}
        \label{eqn:firststage-monotonicity}
        &\Probgiven{ D_i(1) \geq D_i(0) }{\vec X_i} = 1    \\
        \label{eqn:controlfun-iv}
        &\vec X_i^+, \text{ has the property }
        \partialdiff[\mu(\vec X_i)]{\vec X_i^+} = 0 < \partialdiff[D_i(z')]{\vec X_i^+}, \textnormal{ for } z' = 0, 1.
    \end{align}
\end{definition}
Assumption \ref{dfn:controlfun-assumptions}\eqref{eqn:firststage-monotonicity} is the (conditional) monotonicity assumption \citep{imbens1994identification}, which is untestable but accepted in many empirical applications.
Assumption \ref{dfn:controlfun-assumptions}\eqref{eqn:controlfun-iv} is assuming that an instrument exists, which satisfies an exclusion restriction (i.e., not impacting mediator gains $\mu$), and has a non-zero influence on the mediator (i.e., strong first-stage).
The exclusion restriction is untestable, and must be guided by domain-specific knowledge; strength of the first-stage is testable, and must be justified with data by methods common in the IV literature.

Write $K_i$ for the mediator propensity score, as a function of the instrument $\vec X_i^+$ and remaining controls $\vec X_i^-$.
$K_i$ serves as the control function in this setting.
\[ K_i = \Probgiven{D_i = 1}{Z_i, \vec X_i^+, \vec X_i^-} \]

\begin{theorem}
    \label{thm:controlfun}
    If \ref{dfn:controlfun-assumptions}\eqref{eqn:firststage-monotonicity} and \ref{dfn:controlfun-assumptions}\eqref{eqn:controlfun-iv} hold, then the average potential outcomes (and thus, the ADE and AIE) are identified by a control function approach .
    \[ \Egiven{Y_i}{Z_i = z', D_i = d', \vec X_i^-, K_i}
        = \Egiven{Y_i(z', d')}{\vec X_i^-, K_i}
        , \;\; \text{ for } z', d' = 0,1 \]
\end{theorem}
\begin{proof}
    Special case of \citet[Theorem~1]{imbens2009identification}; see \autoref{appendix:controlfun-proof}.
\end{proof}

Assumption \ref{dfn:controlfun-assumptions}\eqref{eqn:firststage-monotonicity} guarantees that mediator $D_i(.)$ can be represented by a selection model \citep{vytlacil2002independence}, and \ref{dfn:controlfun-assumptions}\eqref{eqn:controlfun-iv} pins down a control function to identify the selection model.

\subsection{Relationship to the Roy Model}

Writing here about how the Roy model is solves with this, and the instrument impacts costs of mediator take-up.

\subsection{Simulation Evidence}
