\section{Average Direct and Indirect Effects}
\label{sec:mediation}
CM decomposes causal effects into two channels, through a mediator (indirect effect) and through all other paths (direct effect).
To develop notation, write $Z_i = 0, 1$ for a binary treatment, $D_i = 0, 1$ a binary mediator, and $Y_i$ a continuous outcome.\footnote{
    These are random variables for each.
    This paper exclusively focuses on the binary case.
    See \cite{huber2020direct} for a discussion of CM with continuous treatment and/or mediator, and the assumptions required.
}
%\footnote{
%    Other literatures use different notation.
%    For example, \cite{imai2010identification} write $T_i, M_i, Y_i$ for the randomised treatment, mediator, and outcome, respectively.
%    I use the $Z_i, D_i, Y_i$ instrumental variables notation, more familiar in empirical economics \citep{angrist1996identification}.
%}
\begin{align*}
    D_i &= (1 - Z_i) D_i(0)
        +   Z_i      D_i(1),  \\
    Y_i &= (1 - Z_i) Y_i(0, D_i(0))
        +   Z_i      Y_i(1, D_i(1)).
\end{align*}

%Write $\vec X_i$ for a set of control variables, and assume $Z_i$ is ignorable --- possibly conditional on $\vec X_i$.
Assume treatment $Z_i$ is ignorable.\footnote{
    This assumption can hold conditional on covariates.
    To simplify notation in this section, leave the conditional part unsaid, as it changes no part of the identification framework.
}
\[ Z_i \indep  D_i(z'), Y_i(z, d'), \text{ for } z', z, d' = 0, 1 \]

There are only two average effects which are identified (without additional assumptions).
\begin{enumerate}
    \item The average first-stage refers to the effect of the treatment on mediator, $Z_i$ on $D_i$:
    \[ \Egiven{D_i}{Z_i = 1} - \Egiven{D_i}{Z_i = 0}
        = \E{D_i(1) - D_i(0)}. \]
    It is common in the economics literature to assume that $Z_i$ influences $D_i$ in at most one direction, $\Prob{D_i(0) \leq D_i(1)} = 1$ --- monotonicity \citep{imbens1994identification}.
    I assume mediator monotonicity (and its conditional variant) holds throughout to simplify notation.
    %\footnote{
    %    Assuming monotonicity also brings closer to the IV notation, and has other beneficial implications in this setting (see \autoref{sec:controlfun}).
    %}
    \item The Average Treatment Effect (ATE) refers to the effect of the treatment on outcome, $Z_i$ on $Y_i$, and is also known as the average total effect or intent-to-treat effect in social science settings, or reduced-form effect in the instrumental variables literature:
    \[ \Egiven{Y_i}{Z_i = 1} - \Egiven{Y_i}{Z_i = 0}
        = \E{Y_i(1, D_i(1)) - Y_i(0, D_i(0))}. \]
\end{enumerate}

$Z_i$ affects outcome $Y_i$ directly, and indirectly via the $D_i(Z_i)$ channel, with no reverse causality.
\autoref{fig:scm-model} visualises the design, where the direction arrows denote the causal direction.
CM aims to decompose the ATE of $Z_i$ on $Y_i$ into these two separate pathways:
\begin{align*}
    \text{Average Direct Effect (ADE): } \;\;\;&
        \E{Y_i(1, D_i(Z_i)) - Y_i(0, D_i(Z_i))}, \\
    \text{Average Indirect Effect (AIE): } \;\;\;&
            \E{Y_i(Z_i, D_i(1)) - Y_i(Z_i, D_i(0))}.
\end{align*}

\begin{figure}[h!]
    \centering
    \singlespacing
    \caption{Structural Causal Model for Causal Mediation.}
    \label{fig:scm-model}
    \begin{tikzpicture}
        \node[state,thick,ForestGreen] (mediator) at (0,0) {$D_i$};
        \node[state,thick,blue] (treatment) [left=2.5cm of mediator] {$Z_i$};
        \node[state,thick,red] (outcome) [right=2.5cm of mediator] {$Y_i$};
        % Label Z_i, D, Y_i
        \node[color=ForestGreen] [above=0.1cm of mediator] {Mediator};
        \node[color=blue] [left=0.1cm of treatment] {Treatment};
        \node[text width=0.1cm, color=red] [right=-0.01cm of outcome] {Outcome};
        % Draw the causal arrows
        \path[->, thick] (treatment) edge (mediator);
        \path[->, thick] (mediator) edge (outcome);
        \path[->, thick] (treatment) edge[bend right=45] (outcome);
        % Label direct and indirect effect
        \node[color=orange] [below left=-0.3cm and 0.2cm of mediator] {First-stage};
        \node[color=orange] [below right=-0.3cm and 0.75cm of mediator] {LAIE};
        \node[color=orange] [below=1.2cm of mediator] {ADE};
        % Add in the confounders
        %\node[state,RoyalPurple] (confounderX) [above=1.5cm of mediator] {$\vec{X}$};
        %\path[->,RoyalPurple] (confounderX) edge (mediator);
        %\node[color=RoyalPurple] [left=0.1cm of confounderX] {Observed controls};
        \node[state,thick,dashed,RoyalBlue] (confounderU) [above=0.75cm of outcome] {$\vec U_i$};
        \path[->,thick,dashed,color=RoyalBlue] (confounderU) edge (mediator);
        \path[->,thick,dashed,color=RoyalBlue] (confounderU) edge (outcome);
        %\node[color=RoyalBlue] [right=0.1cm of confounderU] {Unobserved confounder};
    \end{tikzpicture}
    \justify
    \footnotesize
    \textbf{Note}:
    This figure shows the structural causal model behind causal mediation.
    LAIE refers to the AIE (i.e., effect of the mediator $D_i$ on $Y_i$) local to $D_i(Z_i)$ compliers, so that AIE $=$ average first-stage $\times$ LAIE.
    Unobserved confounder $\vec U_i$ represents this paper's focus on the case that $D_i$ is not ignorable, by showing an unobserved confounder.
    \autoref{sec:regression} formally defines $\vec U_i$ in an applied setting.
\end{figure}

Estimating the AIE answers the following question: how much of the causal effect $Z_i$ on $Y_i$ goes through the $D_i$ channel?
If a researcher is studying the health gains of health insurance \citep{finkelstein2008oregon}, and wants to study the role of healthcare usage, the AIE represents how much of the effect comes from using the hospital more often.
Estimating the ADE answers the following equation: how much is left over after accounting for the $D_i$ channel?\footnote{
    In a non-parametric setting it is not necessary that ADE $+$ AIE $=$ ATE.
    See \cite{imai2010identification} for this point in full.
}
For the health insurance example, how much of the health insurance effect is a direct effect, other than increased healthcare usage --- e.g., long-term effects of lower medical debt, or less worry over health shocks.
An instrumental variables approach assumes this direct effect is zero for everyone (the exclusion restriction).
CM is a similar, yet distinct, framework attempting to explicitly model the direct effect, and not assuming it is zero.

The ADE and AIE are not separately identified without further assumptions.

\subsection{Identification of Causal Mediation (CM) Effects}
The conventional approach to estimating direct and indirect effects assumes both $Z_i$ and $D_i$ are ignorable, conditional on a vector of control variables $\vec X_i$.
\begin{definition}
    \label{dfn:seq-ign}
    Sequential Ignorability \citep{imai2010identification}.
    \begin{align}
        \label{eqn:seq-ign-Z}
        Z_i \indep  D_i(z), Y_i(z', d) \;\; &| \;\; \vec X_i,
            &\textnormal{ for } z, z', d = 0, 1 \\
        \label{eqn:seq-ign-D}
        D_i \indep Y_i(z', d) \;\; &| \;\; \vec X_i, Z_i = z', 
            &\textnormal{ for } z', d = 0, 1
    \end{align}
\end{definition}
Sequential ignorability assumes that the initial treatment $Z_i$ is ignorable conditional on $\vec X_i$ (as has already been assumed above).
It then also assumes that, after $Z_i$ is assigned, that $D_i$ is ignorable conditional on $\vec X, Z_i$ (hereafter, mediator ignorability).
If sequential ignorability, \ref{dfn:seq-ign}\eqref{eqn:seq-ign-Z} and \ref{dfn:seq-ign}\eqref{eqn:seq-ign-D}, holds then the ADE and AIE are identified by two-stage mean differences, after conditioning on $\vec X_i$.\footnote{
    In addition, a common support condition for both $Z_i, D_i$ (across $\vec X_i$) is necessary.
    \cite{imai2010identification} show a general identification statement; I show identification in terms of two-stage regression, notation for which is more familiar in economics.
    \aref{appendix:identification} states the \cite{imai2010identification} identification result, and then develops the two-stage regression notation which holds as a consequence of sequential ignorability.
}
\vspace{0.1cm}

\makebox[\textwidth]{\parbox{1.25\textwidth}{
\[ \E[D_i, \vec X_i]{
    \underbrace{\Egiven{Y_i}{Z_i = 1, D_i, \vec X_i} - \Egiven{Y_i}{Z_i = 0, D_i, \vec X_i}}_{\text{Second-stage regression, $Y_i$ on $Z_i$ holding $D_i, \vec X_i$ constant}}}
    = \underbrace{\E{Y_i(1, D_i(Z_i)) - Y_i(0, D_i(Z_i))}}_{\text{Average Direct Effect (ADE)}} \]
\[ \E[Z_i, \vec X_i]{ \underbrace{\Big(
    \Egiven{D_i}{Z_i = 1, \vec X_i} - \Egiven{D_i}{Z_i = 0, \vec X_i} \Big)}_{\text{First-stage regression, $D_i$ on $Z_i$}}
    \times \underbrace{\Big(
    \Egiven{Y_i}{Z_i, D_i = 1, \vec X_i} - \Egiven{Y_i}{Z_i, D_i = 0, \vec X_i} \Big)}_{\text{Second-stage regression, $Y_i$ on $D_i$ holding $Z_i, \vec X_i$ constant}} } \]
\[ = \underbrace{\E{Y_i(Z_i, D_i(1)) - Y_i(Z_i, D_i(0))}}_{\text{Average Indirect Effect (AIE)}} \]
}}
I refer to the estimands on the left-hand side as Causal Mediation (CM) estimands.
These estimands are typically estimated with linear models, with resulting estimates composed from two-stage Ordinary Least Squares (OLS) estimates \citep{imai2010identification}.
While this is the most common approach in the applied literature, I do not assume the linear model.
Linearity assumptions are unnecessary to my analysis; it suffices to note that heterogeneous treatment effects and non-linear confounding would bias OLS estimates of CM estimands in the same manner that is well documented elsewhere (see e.g., \citealt{angrist1998estimating,sloczynski2022interpreting}).
This section focuses on problems that plague CM by selection-on-observables, regardless of estimation method.

\subsection{Non-identification of Causal Mediation (CM) Effects}
Applied researchers often use a natural experiment to study settings where treatment $Z_i$ is ignorable, justifying assumption \ref{dfn:seq-ign}\eqref{eqn:seq-ign-Z}.
Rarely do they also have access to an additional, overlapping natural experiment to isolate random variation in $D_i$ --- to justify mediator ignorability \ref{dfn:seq-ign}\eqref{eqn:seq-ign-D}.
One might consider conventional CM methods in such a setting to learn about the mechanisms behind the causal effect $Z_i$ on $Y_i$.
This approach leads to biased estimates, and contaminates inference regarding direct and indirect effects.

\begin{theorem}
    \label{thm:selection-bias}
    Absent an identification strategy for the mediator, causal mediation estimates are at risk of selection bias.
    If \ref{dfn:seq-ign}\eqref{eqn:seq-ign-Z} holds, and \ref{dfn:seq-ign}\eqref{eqn:seq-ign-D} does not, then CM estimands are contaminated by selection bias and group differences.
    Proof: see \aref{appendix:mediation-bias}.
\end{theorem}
Below I present the relevant selection bias and group difference terms, omitting the conditional on $\vec X_i$ notation for brevity.

\noindent
For the direct effect: CM estimand $=$ ADE $+$ selection bias $+$ group differences.\footnote{
    The bias terms here mirror those in \cite{heckman1998characterizing,angrist2009mostly} for a single $D_i$ on $Y_i$ treatment effect, when $D_i$ is not ignorable:
    \vspace{-0.25cm}
    \[ \Egiven{ Y_i}{D_i =1} - \Egiven{ Y_i}{D_i =0}
        %& = \text{ATT}
        %+ \Big( \Egiven{ Y_i(0)}{D_i =1} - \Egiven{ Y_i(0)}{D_i =0} \Big) \\
        = \text{ATE}
        + \underbrace{\Big( \Egiven{ Y_i(.,0)}{D_i =1} - \Egiven{ Y_i(.,0)}{D_i =0} \Big)}_{
            \text{Selection Bias}}
        + \underbrace{ \Prob{D_i=0} (\text{ATT} - \text{ATU}) }_{
            \text{Group-differences Bias}}. \]
}
\begin{align*}
    & \mathbb E_{D_i} \Big[
        \Egiven{Y_i}{Z_i = 1, D_i} - \Egiven{Y_i}{Z_i = 0, D_i} \Big] \\
    & = \E{Y_i(1, D_i(Z_i)) - Y_i(0, D_i(Z_i))} \\
    & \;\;\;\; + \mathbb E_{D_i = d'} \Big[
        \Egiven{Y_i(0, D_i(Z_i))}{D_i(1) = d'} 
        - \Egiven{Y_i(0, D_i(Z_i))}{D_i(0) = d'} \Big] \\
    & \;\;\;\; + \E[D_i = d']{
        \Big(1 - \Prob{D_i(1) = d'} \Big)
        \left( \begin{aligned}
            &\Egiven{Y_i(1, D_i(Z_i)) - Y_i(0, D_i(Z_i))}{D_i(1) = 1-d'} \\ 
            &  - \Egiven{Y_i(1, D_i(Z_i)) - Y_i(0, D_i(Z_i))}{D_i(1) = d'}
            \end{aligned} \right) }
\end{align*}

\noindent
For the indirect effect: CM estimand $=$ AIE $+$ selection bias $+$ group differences.
\begin{align*}
    &\E[Z_i]{
        \Big( \Egiven{D_i}{Z_i = 1} - \Egiven{D_i}{Z_i = 0} \Big) \times
        \Big( \Egiven{Y_i}{Z_i, D_i = 1} - \Egiven{Y_i}{Z_i, D_i = 0} \Big) } \\
    & = \E{Y_i(Z_i, D_i(1)) - Y_i(Z_i, D_i(0))} \\
    & \;\;\;\; + \Prob{D_i(1) = 1, D_i(0) = 0} \Big(
        \Egiven{Y_i(Z_i, 0)}{D_i = 1} - \Egiven{Y_i(Z_i, 0)}{D_i = 0} \Big) \\
    & \;\;\;\; + \Prob{D_i(1) = 1, D_i(0) = 0} \times \\
    & \;\;\;\; \;\; \left[ \begin{aligned}
        &\Big( 1 - \Prob{D_i=1} \Big)
        \left( \begin{aligned}
            &\Egiven{Y_i(Z_i, 1) - Y_i(Z_i, 0)}{D_i = 1} \\ 
            &  - \Egiven{Y_i(Z_i, 1) - Y_i(Z_i, 0)}{D_i = 0}
        \end{aligned} \right) \\
        &- \left( \frac{1 - \Prob{D_i(1) = 1, D_i(0) = 0} }{
            \Prob{D_i(1) = 1, D_i(0) = 0}} \right)
        \left( \begin{aligned}
            &\Egiven{Y_i(Z_i, 1) - Y_i(Z_i, 0)}{D_i(1) = 0 \text{ or } D_i(0)=1} \\ 
            &  - \E{Y_i(Z_i, 1) - Y_i(Z_i, 0)}
        \end{aligned} \right)
    \end{aligned} \right]
\end{align*}

The selection bias terms come from systematic differences between the groups taking or refusing the mediator ($D_i = 1$ versus $D_i = 0$), differences not fully unexplained by $\vec X_i$.
These selection bias terms would equal zero if the mediator had been ignorable \ref{dfn:seq-ign}\eqref{eqn:seq-ign-D}, but do not necessarily average to zero if not.%\footnote{
%    The selection-on-observables approach could, instead, focus on the average effect on treated populations --- as do \cite{keele2015identifying}.
%    This runs into a problem of comparisons: CM estimates would give average effects on different treated groups.
%    The CM estimand for the ADE on treated gives the ADE local to the $Z_i = 1$ treated group, and for the ADE local to compliers with $D_i = 1$.
%    In this way, these ADE and AIE on treated terms are not comparable to each other, so I focus on the true averages to avoid these misaligned comparisons.
%}

The group differences represent the fact that a matching approach gives an average effect on the treated group and, when selection-on-observables does not hold, this is systematically different from the average effect \citep{heckman1998characterizing}.
These terms are a non-parametric framing of the bias from controlling for intermediate outcomes, previously studied only in a linear setting (i.e., bad controls in \citealt{cinelli2024crash}, or M-bias in \citealt{ding2015adjust}).

The AIE group differences term is longer, because the indirect effect is comprised of the effect of $D_i$ local to $D_i(Z_i)$ compliers.
\begin{align*}
    \text{AIE}
    &= \E{Y_i(Z_i, D_i(1)) - Y_i(Z_i, D_i(0))} \\
    &= \E{D_i(1) - D_i(0)} \; 
        \underbrace{\Egiven{Y_i(Z_i, 1) - Y_i(Z_i, 0)}{D_i(0)=0, D_i(1)=1}}_{
            \text{Average $D_i$ on $Y_i$ effect among $D_i(Z_i)$ compliers}
        }
\end{align*}
It is important to acknowledge the mediator compliers here, because the AIE is the treatment effect going through the $D_i(Z_i)$ channel, thus only refers to individuals pushed into mediator $D_i$ by initial treatment $Z_i$.
If we had been using a population average effect for $D_i$ on $Y_i$, then this is losing focus on the definition of the AIE; it is not about the causal effect $D_i$ on $Y_i$, it is about the causal effect $D_i(Z_i)$ on $Y_i$.

The group difference bias term arises because the selection-on-observables approach assumes that this complier average effect is equal to the population average effect, which does not hold true if the mediator is not ignorable.
This distinction between average effects and complier average effects in the AIE is skipped over by the ``controlled effect'' definitions of \cite{pearl2003direct}.
