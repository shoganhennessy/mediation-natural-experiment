\section{Direct and Indirect Effects}
\label{sec:mediation}
Causal mediation decomposes causal effects into two channels, through a mediator (indirect effect) and through all other paths (direct effect).
To develop notation for direct and indirect effects, write $Z_i$ for an exogenous binary treatment, $D_i$ a binary mediator, and $Y_i$ an outcome for individuals $i = 1, \hdots, n$.\footnote{
    Other literatures use different notation.
    For example, \cite{imai2010identification} write $T_i, M_i, Y_i$ for the randomised treatment, mediator, and outcome, respectively.
    I use the $Z_i, D_i, Y_i$ instrumental variables notation, more familiar in empirical economics \citep{angrist1996identification}.
}
The outcomes are a sum of their potential outcomes.\footnote{
    This paper exclusively focuses on the binary case.
    See \cite{huber2020direct} for a discussion of CM with continuous treatment and/or mediator, and the assumptions required.
}
\begin{align*}
    D_i &= Z_i       D_i(1)
        + (1 - Z_i) D_i(0),  \\
    Y_i &= Z_i       Y_i(1, D_i(1))
        + (1 - Z_i) Y_i(0, D_i(0)).
\end{align*}

%Write $\vec X_i$ for a set of control variables, and assume $Z_i$ is ignorable --- possibly conditional on $\vec X_i$.
Assume $Z_i$ is ignorable.\footnote{
    This assumption can hold conditional on covariates.
    To simplify notation in this section, leave the conditional part unsaid, as it changes no part of the identification framework.
}
\[ Z_i \indep  D_i(z), Y_i(z', d), \textnormal{ for } z, z', d = 0, 1 \]

There are only two average effects which are identified (without additional assumptions).
\begin{enumerate}
    \item The average first-stage refers to the effect of the treatment on mediator, $Z \to D$.
    \[ \Egiven{D_i}{Z_i = 1} - \Egiven{D_i}{Z_i = 0}
        = \E{D_i(1) - D_i(0)} \]
    It common in the economics literature to assume that $Z$ influences $D$ in at most one direction, $\Prob{D_i(1) \geq D_i(0)} = 1$ --- monotonicity \citep{imbens1994identification}.
    I assume monotonicity (and its conditional variant) holds through-out to simplify notation.\footnote{
        Assuming monotonicity also brings closer to the IV notation, and has other beneficial implications in this setting (see \autoref{sec:controlfun}).
    }
    \item The reduced-form effect refers to the effect of the treatment on outcome, $Z \to Y$, and is also known as the intent-to-treat effect in experimental settings, or total effect in causal mediation literature.
    \[ \Egiven{Y_i}{Z_i = 1} - \Egiven{Y_i}{Z_i = 0}
        = \E{Y_i(1, D_i(1)) - Y_i(0, D_i(0))} \]
\end{enumerate}

\begin{figure}[h!]
    \centering
    \singlespacing
    \caption{Structural Causal Model for Causal Mediation.}
    \label{fig:scm-model}
    \begin{tikzpicture}
        \node[state,ForestGreen] (treatment) at (0,0) {$D$};
        \node[state,blue] (instrument) [left=2.5cm of treatment] {$Z$};
        \node[state,red] (outcome) [right=2.5cm of treatment] {$Y$};
        % Label Z, D, Y
        \node[color=ForestGreen] [above=0.1cm of treatment] (mediator) {Mediator};
        \node[color=blue] [left=0.1cm of instrument] {Treatment};
        \node[text width=0.1cm, color=red] [right=-0.01cm of outcome] {Outcome};
        % Draw the causal arrows
        \path[->, thick] (instrument) edge (treatment);
        \path[->, thick] (treatment) edge (outcome);
        \path[->, thick] (instrument) edge[bend right=45] (outcome);
        % Label direct and indirect effect
        \node[color=orange] [below left=-0.2cm and 0.2cm of treatment] {First-stage};
        \node[color=orange] [below right=-0.2cm and 0.5cm of treatment] {LAIE};
        \node[color=orange] [below=1.2cm of treatment] {ADE};
        % Add in the confounders
        %\node[state,RoyalPurple] (confounderX) [above=1.5cm of treatment] {$\vec{X}$};
        %\path[->,RoyalPurple] (confounderX) edge (treatment);
        %\node[color=RoyalPurple] [left=0.1cm of confounderX] {Observed controls};
        \node[state,dashed,RoyalBlue] (confounderU) [above=0.75cm of outcome] {$U$};
        \path[->,dashed,color=RoyalBlue] (confounderU) edge (treatment);
        \path[->,dashed,color=RoyalBlue] (confounderU) edge (outcome);
        %\node[color=RoyalBlue] [right=0.1cm of confounderU] {Unobserved confounder};
    \end{tikzpicture}
    \justify
    \footnotesize
    \textbf{Note}:
    This figures shows the structural causal model behind causal mediation.
    LAIE refers to the AIE (i.e., effect of the mediator $D \to Y$) local to $Z$ compliers, so that AIE $=$ average first-stage $\times$ LAIE.
    Unobserved confounder $U$ represents this paper's focus on the case that $D_i$ is not ignorable, by showing an implied unobserved confounder.
    \autoref{sec:regression} formally defines $U$ in this set-up.
\end{figure}

In this setting, $Z_i$ affects outcome $Y_i$ directly, and indirectly via the $D_i(Z_i)$ channel, with no reverse causality.
% The framework is general to any (conditionally) randomly assigned $Z_i$, intermiedate mediator $D_i$, and outcome $Y_i$.
%This section focuses on binary $Z_i, D_i = 0,1$, and not the continuous versions, to simplify the causal framework.%\footnote{
    %\autoref{appendix:continuous} relates the causal framework to continuous $Z_i,D_i$, which has little conceptual differences.}
\autoref{fig:scm-model} visualises the design, where the direction arrows denote the causal direction (and no reverse causality).
CM aims to decompose the reduced form effect of $Z \to Y$ into these two separate pathways.
\begin{align*}
    \text{Average Indirect Effect (AIE), } D(Z) \to Y: \;\;\;&
        \E{Y_i(Z_i, D_i(1)) - Y_i(Z_i, D_i(0))} \\
    \text{Average Direct Effect (ADE), } Z \to Y: \;\;\;&
        \E{Y_i(1, D_i(Z_i)) - Y_i(0, D_i(Z_i))}
\end{align*}
Estimating the AIE answers the following question: how much of the causal effect $Z \to Y$ goes through the $D$ channel?
If a researcher is studying the income effect of a man being randomly drafted into the US military to serve in the Vietnam war, and is interested in military service as the mediator, the AIE represents how much of the effect comes from military service.
In an instrumental variables application, this direct effect is assumed to be zero for everyone --- that the draft had no other effects (i.e., the exclusion restriction, see \citet{angrist1990lifetime,angrist2011schooling}).
CM is a different framework attempting to explicitly model the direct effect, not assuming the ADE is zero.
Estimating the ADE answers the following equation: how much is left over after accounting for the $D$ channel?\footnote{
    In a non-parametric setting it is not necessary that the ADE and AIE sum to the total effect.
    See \cite{imai2010identification} for this point in full.
}
For the military draft example, how much of the effect of random conscription is a direct effect --- e.g., from higher education draft deferment, and other actions to dodge the draft?
These CM effects are not separately identified without further assumptions.

\subsection{Identifying Causal Mediation (CM) Effects}
The conventional approach to estimating direct and indirect effects assumes both $Z_i$ and $D_i$ are ignorable, conditional on a set of control variables $\vec X_i$.
\begin{definition}
    \label{dfn:seq-ign}
    Sequential Ignorability \citep{imai2010identification}.
    \begin{align}
        \label{eqn:seq-ign-Z}
        Z_i \indep  D_i(z), Y_i(z', d) \;\; &| \;\; \vec X_i,
            &\textnormal{ for } z, z', d = 0, 1 \\
        \label{eqn:seq-ign-D}
        D_i \indep Y_i(z', d) \;\; &| \;\; \vec X_i, Z_i = z', 
            &\textnormal{ for } z', d = 0, 1
    \end{align}
\end{definition}
Sequential ignorability assumes that the initial treatment $Z_i$ is assigned randomly, conditional on $\vec X_i$.
It then also assumes that, after $Z_i$ is assigned, that $D_i$ is assigned randomly conditional $\vec X_i, Z_i$.
If sequential ignorability, \ref{dfn:seq-ign}\eqref{eqn:seq-ign-Z} and \ref{dfn:seq-ign}\eqref{eqn:seq-ign-D}, holds then the ADE and AIE are identified by two-stage mean differences, after conditioning on $\vec X_i$.\footnote{
    \cite{imai2010identification} show a general identification statement; I show identification in terms of two-stage regression, notation for which is more familiar in economics.
    This reasoning is in line with G-computation reasoning \citep{robins1986g};
    \autoref{appendix:identification} states the \cite{imai2010identification} identification result, and then develops the two-stage regression notation which holds as a consequence of sequential ignorability.
}
\vspace{0.1cm}

\makebox[\textwidth]{\parbox{1.25\textwidth}{
\[ \E[D_i = d', \vec X_i]{
    \underbrace{\Egiven{Y_i}{Z_i = 1, D_i = d', \vec X_i} - \Egiven{Y_i}{Z_i = 0, D_i = d', \vec X_i}}_{\text{Second-stage regression, $Y_i$ on $Z_i$ holding $D_i$ constant}}}
    = \underbrace{\E{Y_i(1, D_i(Z_i)) - Y_i(0, D_i(Z_i))}}_{\text{Average Direct Effect (ADE)}} \]
\[ \E[Z_i = z', \vec X_i]{ \underbrace{\Big(
    \Egiven{D_i}{Z_i = 1, \vec X_i} - \Egiven{D_i}{Z_i = 0, \vec X_i} \Big)}_{\text{First-stage regression, $D_i$ on $Z_i$}}
    \times \underbrace{\Big(
    \Egiven{Y_i}{Z_i = z', D_i = 1, \vec X_i} - \Egiven{Y_i}{Z_i = z', D_i = 0, \vec X_i} \Big)}_{\text{Second-stage regression, $Y_i$ on $D_i$ holding $Z_i$ constant}} } \]
\[ = \underbrace{\E{Y_i(Z_i, D_i(1)) - Y_i(Z_i, D_i(0))}}_{\text{Average Indirect Effect (AIE)}} \]
}}
I refer to the estimands on the left-hand side as Causal Mediation (CM) estimands.
These estimands are typically estimated with linear models, with resulting estimates composed from two-stage Ordinary Least Squares (OLS) estimates \citep{imai2010identification}.
%\begin{align*}
%    D_i &= \phi + \pi Z_i
%        + \vec \psi_1' \vec X_i+ \eta_i \\
%    Y_i &= \alpha + \beta D_i + \gamma Z_i + \delta Z_i D_i
%        + \vec \psi_2' \vec X_i + \varepsilon_i
%\end{align*}
%And so the CM estimands are composed from OLS estimates,
%$\hat \gamma + \hat\delta \E{D_i}$ for the Average Direct Effect (ADE) and
%$\hat\pi \left(\hat \beta + \E{Z_i} \hat \delta \right)$ for the average indirect effect (AIE).
While this is the most common approach in the applied literature, I do not assume the linear model.
% of this problem as it assumes homogenous treatment effects and linear confounding.
Linearity assumptions are unnecessary to my analysis; it suffices to note that heterogeneous treatment effects and non-linear confounding would bias OLS estimates of CM estimands in the same manner that is well documented elsewhere (see e.g., \citealt{angrist1998estimating,sloczynski2022interpreting}).
This section focuses on problems that plague CM in practice, regardless of estimation method.
% As such, I focus my work on non-parametric identification, and employ semi- and non-parametric estimation methods in my empirical analysis  whenever possible to avoid these problems.

\subsection{Bias in Causal Mediation (CM) Estimates}
Applied research may use a natural experiment to justify the treatment $Z_i$ is ignorable, justifying assumption \ref{dfn:seq-ign}\eqref{eqn:seq-ign-Z}.
Rarely does research relying on a quasi-experimental research design employ an additional, overlapping identification design for $D_i$ to justify assumption \ref{dfn:seq-ign}\eqref{eqn:seq-ign-D} as part of the analysis.
One might consider conventional CM methods in such a setting to learn about the mechanisms behind the causal effect under study.
This approach leads to biased estimates, and contaminates inference regarding direct and indirect effects.%\footnote{
    %    \cite{imai2013experimental} call attention to the need for a separate research design to isolate causal effects of $D_i$ in randomised controlled trials; \autoref{appendix:mediation-review} overviews literature, finding many papers that employ mediation methods with a research design for $Z_i$, but not for $D_i$.
    %}

\begin{theorem}
    \label{thm:selection-bias}
    Absent an identification strategy for the mediator, causal mediation estimates are at risk of selection bias.
    Suppose \ref{dfn:seq-ign}\eqref{eqn:seq-ign-Z} holds, but \ref{dfn:seq-ign}\eqref{eqn:seq-ign-D} does not.
    Then CM estimands are contaminated by selection bias and group differences.
\end{theorem}
\begin{proof}
    See \autoref{appendix:mediation-bias} for the proof.
    Below I present the relevant selection bias and group difference terms, omitting the conditional on $\vec X_i$ notation for brevity.
\end{proof}

\noindent
For the direct effect: CM estimand $=$ ADE $+$ selection bias $+$ group differences.
\begin{align*}
    & \mathbb E_{D_i = d'} \Big[
        \Egiven{Y_i}{Z_i = 1, D_i = d'} - \Egiven{Y_i}{Z_i = 0, D_i = d'} \Big] \\
    & = \E{Y_i(1, D_i(Z_i)) - Y_i(0, D_i(Z_i))} \\
    & \;\;\;\; + \mathbb E_{D_i = d'} \Big[
        \Egiven{Y_i(0, D_i(Z_i))}{D_i(1) = d'} 
        - \Egiven{Y_i(0, D_i(Z_i))}{D_i(0) = d'} \Big] \\
    & \;\;\;\; + \E[D_i = d']{
        \Big(1 - \Prob{D_i(1) = d'} \Big)
        \left( \begin{aligned}
            &\Egiven{Y_i(1, D_i(Z_i)) - Y_i(0, D_i(Z_i))}{D_i(1) = 1-d'} \\ 
            &  - \Egiven{Y_i(1, D_i(Z_i)) - Y_i(0, D_i(Z_i))}{D_i(1) = d'}
            \end{aligned} \right) }
\end{align*}

\noindent
For the indirect effect: CM estimand $=$ AIE $+$ selection bias $+$ group differences.\footnote{
    The bias terms here mirror those in \cite{heckman1998characterizing,angrist2009mostly} for a one dimensional $D\to Y$ treatment effect, when $D_i$ is not ignorable.
    \[ \Egiven{ Y_i}{D_i =1} - \Egiven{ Y_i}{D_i =0}
        %& = \text{ATT}
        %+ \Big( \Egiven{ Y_i(0)}{D_i =1} - \Egiven{ Y_i(0)}{D_i =0} \Big) \\
        = \text{ATE}
        + \underbrace{\Big( \Egiven{ Y_i(0)}{D_i =1} - \Egiven{ Y_i(0)}{D_i =0} \Big)}_{
            \text{Selection Bias}}
        + \underbrace{ \Prob{D_i=0} (\text{ATT}- \text{ATU}) }_{
            \text{Group-differences Bias}} \]
}
\begin{align*}
    &\E[Z_i = z']{
        \Big( \Egiven{D_i}{Z_i = 1} - \Egiven{D_i}{Z_i = 0} \Big) \times
        \Big( \Egiven{Y_i}{Z_i = z', D_i = 1} - \Egiven{Y_i}{Z_i = z', D_i = 0} \Big) } \\
    & = \E{Y_i(Z_i, D_i(1)) - Y_i(Z_i, D_i(0))} \\
    & \;\;\;\; + \Prob{D_i(1) = 1, D_i(0) = 0} \Big(
        \Egiven{Y_i(Z_i, 0)}{D_i = 1} - \Egiven{Y_i(Z_i, 0)}{D_i = 0} \Big) \\
    & \;\;\;\; + \Prob{D_i(1) = 1, D_i(0) = 0} \times \\
    & \;\;\;\; \;\; \left[ \begin{aligned}
        &\Big( 1 - \Prob{D_i=1} \Big)
        \left( \begin{aligned}
            &\Egiven{Y_i(Z_i, 1) - Y_i(Z_i, 0)}{D_i = 1} \\ 
            &  - \Egiven{Y_i(Z_i, 1) - Y_i(Z_i, 0)}{D_i = 0}
        \end{aligned} \right) \\
        &- \left( \frac{1 - \Prob{D_i(1) = 1, D_i(0) = 0} }{
            \Prob{D_i(1) = 1, D_i(0) = 0}} \right)
        \left( \begin{aligned}
            &\Egiven{Y_i(Z_i, 1) - Y_i(Z_i, 0)}{D_i(1) = 0 \text{ or } D_i(0)=1} \\ 
            &  - \E{Y_i(Z_i, 1) - Y_i(Z_i, 0)}
        \end{aligned} \right)
    \end{aligned} \right]
\end{align*}

The selection bias terms come from systematic differences between groups who do and do not take the mediator ($D_i = 0$ versus $D_i = 1$), differences not fully unexplained by $\vec X_i$.
These selection bias terms would equal to zero if the mediator was ignorable \ref{dfn:seq-ign}\eqref{eqn:seq-ign-D}, but do not necessarily average to zero if not.
The group differences represent the fact that a matching estimator gives an average effect on the treated group and, when selection-on-observables does not hold, this is systematically different from the average effect \citep{heckman1998characterizing}.\footnote{
    The group differences term is longer for the AIE estimate, because the indirect effect is comprised from the effect of $D_i$ local to $Z_i$ compliers; a matching estimator gets the average effect on treated, and the longer term adjusts for differences with the complier average effect.
}$^{,}$\footnote{
    The selection-on-observables approach could, instead, focus on the average effect on treated populations (as do \citealt{keele2015identifying}).
    This runs into a problem of comparisons: CM estimates would give average effects on different treated groups.
    The CM estimand for the ADE on treated gives the ADE local to the $Z_i = 1$ treated group, and local to the $D_i = 1$ group for the AIE.
    In this way, these ADE and AIE on treated terms are not comparable to each other, so I focus on the true averages to avoid these misaligned comparisons.
}
The group differences term is a non-parametric framing of the bias from controlling for intermediate outcomes, previously studied only in a linear setting (i.e., bad controls in \citealt{cinelli2024crash}, or M-bias in \citealt{ding2015adjust}).
