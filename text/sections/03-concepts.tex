\section{Direct and Indirect Effects}
\label{sec:concepts}
Causal mediation decomposes causal effects into two channels, through a mediator (indirect effect) and through all other paths (direct effect).

%\textbf{To add: Why do we even want these effects?}

To develop notation for direct and indirect effects, write $Z_i$ for an exogenous binary variable, $D_i$ an intermediary outcome, and $Y_i$ an outcome for individuals $i = 1, \hdots, n$.
The outcomes are a sum of their potential outcomes:
\begin{align*}
    D_i &= Z_i       D_i(1)
        + (1 - Z_i) D_i(0),  \\
    Y_i &= Z_i       Y_i(1, D_i(1))
        + (1 - Z_i) Y_i(0, D_i(0))
\end{align*}
$Z_i$ affects outcome $Y_i$ directly, and indirectly via the $D_i(Z_i)$ channel, with no reverse causality.
The framework is general to any (conditionally) randomly assigned $Z_i$, selected mediator $D_i$, and outcome $Y_i$.
For the analysis in \autoref{sec:data}, $Z_i$ is the Ed PGI, $D_i$ a measure of education year, and $Y_i$ (log) later-life earnings.
Note that     $Z_i, D_i$ are continuous measures in HRS Data, but this section focuses on binary $Z_i, D_i = 0,1$ to simplify the causal framework.\footnote{
    Continuous analogues of the following are extensions of the binary case, and will be included in work to follow.
    %\autoref{appendix:continuous} relates the causal framework to continuous $Z_i,D_i$, which has little conceptual differences.
}
\autoref{fig:scm-model} visualises the design, where the direction arrows denote the causal direction (and no reverse causality).

\begin{figure}[h!]
    \centering
    \singlespacing
    \caption{Structural Causal Model for Direct and Indirect Effects of Genetics and Education.}
    \label{fig:scm-model}
    \begin{tikzpicture}
        \node[state,ForestGreen] (treatment) at (0,0) {$D$};
        \node[state,blue] (instrument) [left=2cm of treatment] {$Z$};
        \node[state,red] (outcome) [right=2cm of treatment] {$Y$};
        \path[->] (instrument) edge (treatment);
        \path[->] (treatment) edge (outcome);
        % Label the nodes with econ examples
        \node[text width=2cm, color=blue] [left=0.1cm of instrument] {Ed PGI};
        \node[text width=0.1cm, color=red] [right=-0.01cm of outcome] {Earnings};
        \node[text width=1.5cm, color=ForestGreen] [above=0.5cm of treatment] {Education};
        % Add in direct effects.
        \path[->] (instrument) edge[bend right=45] (outcome);
        % Label the causal effects
        \node[text height=1cm, color=orange] [left=0.5cm of outcome] {LATE};
        \node[text height=1cm, color=orange] [right=0.175cm of instrument] {First-stage};
        \node[text height=1cm, color=orange] [below=0.25cm of treatment] {Direct Effect};
        % Add in the confounders
        \node[state,RoyalPurple] (confounderX) [above left=1.5cm of treatment] {$\vec{X}$};
        \path[->,RoyalPurple] (confounderX) edge (treatment);
        \node[text width=1.5cm, color=RoyalPurple] [left=0.1cm of confounderX] {Observed controls};
        \node[state,RoyalBlue] (confounderU) [above right=1.5cm of treatment] {$U$};
        \path[->,dotted,color=RoyalBlue] (confounderU) edge (treatment);
        \node[text width=2.5cm, color=RoyalBlue] [right=0.1cm of confounderU] {Unobserved confounder};
    \end{tikzpicture}
    \justify
    \footnotesize
    \textbf{Note}:
    This figures shows the structural causal model for decomposing the direct and indirect effects of genetics and education attainment.
\end{figure}

Assuming that Ed PGI $Z_i$ is randomly assigned (or conditionally so, see \autoref{sec:data-genescore}), then there are only two average effects which are identified.
The first-stage effect refers to the effect of the Ed PGI on education, $Z \to D$.
\[ \Egiven{D_i}{Z_i = 1} - \Egiven{D_i}{Z_i = 0}
    = \E{D_i(1) - D_i(0)} \]
It common in the economics literature to assume that $Z$ influences $D$ in at most one direction, $\Prob{D_i(1) \geq D_i(0)} = 1$ --- monotonicity \citep{imbens1994identification}.
I assume monotonicity (and its conditional variant) holds through-out, as it brings the mediation notation closer to the IV literature from labour economics.\footnote{
    Monotonicity has other beneficial implications in this setting, as shown in \autoref{sec:selection-model}.
} 

The reduced-form effect refers to the effect of the Ed PGI on earnings, $Z \to Y$, and is also known as the intent-to-treat effect in experimental settings, or total effect in causal mediation literature.
\[ \Egiven{Y_i}{Z_i = 1} - \Egiven{Y_i}{Z_i = 0}
    = \E{Y_i(1, D_i(1)) - Y_i(0, D_i(0))} \]

On the other hand, mediation aims to decompose the reduced form effect of $Z \to Y$ into two separate pathways: indirectly through $D$, and directly absent $D$.
\begin{align*}
    \text{Indirect Effect, } D(Z) \to Y: \;\;\;&
        \E{Y_i(Z_i, D_i(1)) - Y_i(Z_i, D_i(0))} \\
    \text{Direct Effect, } Z \to Y: \;\;\;&
        \E{Y_i(1, D_i(Z_i)) - Y_i(0, D_i(Z_i))}
\end{align*}
These effects are not separately identified without further assumptions.

\subsection{Causal Mediation Estimates}
The conventional approach to estimating direct and indirect effects assumes both $Z_i$ and $D_i$ are conditionally ignorable.
\begin{definition}
    \label{dfn:seq-ign}
    Sequential Ignorability \citep{imai2010identification}.
    \begin{align}
        \label{eqn:seq-ign-Z}
        Z_i \indep  D_i(z), Y_i(z', d) \;\; &| \;\; \vec X_i,
            &\textnormal{ for } z, z', d = 0, 1 \\
        \label{eqn:seq-ign-D}
        D_i \indep  Y_i(z', d) \;\; &| \;\; \vec X_i, Z_i = z', 
            &\textnormal{ for } z', d = 0, 1
    \end{align}
\end{definition}
If \ref{dfn:seq-ign}\eqref{eqn:seq-ign-Z} and \ref{dfn:seq-ign}\eqref{eqn:seq-ign-D} hold, then the direct and indirect effects are identified by two-stage mean differences, after conditioning on $\vec X_i$:\footnote{
    \cite{imai2010identification} show a general identification statement; I show identification in terms of two-stage regression, which is more familiar in economics.
    This reasoning is in line with G-computation reasoning \citep{robins1986g};
    \autoref{appendix:identification} states the \cite{imai2010identification} identification result, and then develops the two-stage regression notation which holds as a consequence of sequential ignorability.
}

\makebox[\textwidth]{\parbox{1.25\textwidth}{
\[ \E[D_i, \vec X_i]{
    \underbrace{\Egiven{Y_i}{Z_i = 1, D_i, \vec X_i} - \Egiven{Y_i}{Z_i = 0, D_i, \vec X_i}}_{\text{Second-stage regression, $Y_i$ on $Z_i$ holding $D_i$ constant}}}
    = \underbrace{\E{Y_i(1, D_i(Z_i)) - Y_i(0, D_i(Z_i))}}_{\text{Average Direct effect}} \]
\[ \E[Z_i, \vec X_i]{ \underbrace{\Big(
    \Egiven{D_i}{Z_i = 1, \vec X_i} - \Egiven{D_i}{Z_i = 0, \vec X_i} \Big)}_{\text{First-stage regression, $D_i$ on $Z_i$}}
    \times \underbrace{\Big(
    \Egiven{Y_i}{Z_i, D_i = 1, \vec X_i} - \Egiven{Y_i}{Z_i, D_i = 0, \vec X_i} \Big)}_{\text{Second-stage regression, $Y_i$ on $D_i$ holding $Z_i$ constant}} } \]
\[ = \underbrace{\E{Y_i(Z_i, D_i(1)) - Y_i(Z_i, D_i(0))}}_{\text{Average Indirect effect}} \]
}}

These estimands are typically estimated with linear models \citep{imai2010identification}:
\begin{align*}
    D_i &= \phi + \pi Z_i
        + \vec \psi_1' \vec X_i+ \eta_i \\
    Y_i &= \alpha + \beta D_i + \gamma Z_i + \delta Z_i D_i
        + \vec \psi_2' \vec X_i + \varepsilon_i
\end{align*}
And so the direct and indirect effects are composed from OLS estimates,
$\hat \gamma + \hat\delta \E{D_i}$ for the direct effect and
$\hat\pi \left(\hat \beta + \E{Z_i} \hat \delta \right)$ for the indirect effect.
While this is the most common approach in the applied literature, I do not focus on the linear formulation of this problem as it assumes homogenous treatment effects and linear confounding.
These assumptions are unnecessary to my analysis; it suffices to note that heterogeneous treatment effects and non-linear confounding would bias OLS estimates of direct and indirect effects in the same manner that is well documented elsewhere (see e.g., \citealt{angrist1998estimating,sloczynski2022interpreting}).
I focus on fundamental problems that plague causal mediation methods in practice, regardless of estimation method.
As such, I focus my work on non-parametric identification, and employ semi- and non-parametric estimation methods in my empirical analysis  whenever possible to avoid these problems.

\subsection{Selection Bias in Causal Mediation Estimates}
Mediation methods are the main method that researchers then answer the following question: how did $Z$ lead to a causal effect on $Y$, and through which channels?
In observational work this may include a natural experiment that quasi-randomly assigns $Z_i$ to individuals, regardless of their preferences or selection patterns --- i.e., justifying assumption \ref{dfn:seq-ign}\eqref{eqn:seq-ign-Z}.
Rarely does observational research employ an additional, overlapping identification design for $D_i$ as part of the analysis, and instead they employ mediation methods by assuming this $D_i$ is ignorable given observed covariates $\vec X_i$.\footnote{
    \cite{imai2013experimental} call attention to the need for a separate research design to isolate causal effects of $D_i$ in randomised controlled trials; \autoref{appendix:mediation-review} overviews literature, finding many papers that employ mediation methods with a research design for $Z_i$, but not for $D_i$.
}
This approach leads to biased estimates, and contaminates inference regarding direct and indirect effects (in the same manner as \citealt{heckman1998characterizing}).

\begin{theorem}
    \label{thm:selection-bias}
    Absent an identification strategy for the mediator, causal mediation estimates are at risk of selection bias.
    Suppose \ref{dfn:seq-ign}\eqref{eqn:seq-ign-Z} holds, but \ref{dfn:seq-ign}\eqref{eqn:seq-ign-D} does not.
    Then causal mediation estimates are contaminated by selection bias terms, and group differences terms.
\end{theorem}
\begin{proof}
    See \autoref{appendix:selection-bias} for the extended proof.
\end{proof}
Below I present the relevant selection bias and group difference terms, omitting the conditional on $\vec X_i$ notation for brevity.
These selection bias terms would be equal to zero if the mediator was conditional ignorable \eqref{eqn:seq-ign-D}, but do not necessarily average to zero if not.

For the average direct effect: CM estimate $=$ ADE $+$ selection bias $+$ group differences.
\vspace{-0.5cm}
\begin{align*}
    & \mathbb E_{D_i} \Big[
        \Egiven{Y_i}{Z_i = 1, D_i} - \Egiven{Y_i}{Z_i = 0, D_i} \Big] \\
    & = \E{Y_i(1, D_i(Z_i)) - Y_i(0, D_i(Z_i))} \\
    & \;\;\;\; + \mathbb E_{D_i} \Big[
        \Egiven{Y_i(0, D_i(Z_i))}{D_i(1) = d} 
        - \Egiven{Y_i(0, D_i(Z_i))}{D_i(0) = d} \Big] \\
    & \;\;\;\; + \E[D_i ]{
        \Big(1 - \Prob{D_i(1) = d} \Big)
        \left( \begin{aligned}
            &\Egiven{Y_i(1, D_i(Z_i)) - Y_i(0, D_i(Z_i))}{D_i(1) = d} \\ 
            &  - \Egiven{Y_i(1, D_i(Z_i)) - Y_i(0, D_i(Z_i))}{D_i(0) = 1- d}
            \end{aligned} \right) }
\end{align*}

For the average indirect effect: CM estimate $=$ AIE $+$ selection bias $+$ group differences.
\vspace{-0.5cm}
\begin{align*}
    &\E[Z_i]{
        \Big( \Egiven{D_i}{Z_i = 1} - \Egiven{D_i}{Z_i = 0} \Big) \times
        \Big( \Egiven{Y_i}{Z_i, D_i = 1} - \Egiven{Y_i}{Z_i, D_i = 0} \Big) } \\
    & = \E{Y_i(Z_i, D_i(1)) - Y_i(Z_i, D_i(0))} \\
    & \;\;\;\; + \Prob{D_i(1) = 1, D_i(0) = 0} \Big(
        \Egiven{Y_i(Z_i, 0)}{D_i = 1} - \Egiven{Y_i(Z_i, 0)}{D_i = 0} \Big) \\
    & \;\;\;\; + \Prob{D_i(1) = 1, D_i(0) = 0} \times \\
    & \;\;\;\; \;\; \left[ \begin{aligned}
        &\Big( 1 - \Prob{D_i=1} \Big)
        \left( \begin{aligned}
            &\Egiven{Y_i(Z_i, 1) - Y_i(Z_i, 0)}{D_i = 1} \\ 
            &  - \Egiven{Y_i(Z_i, 1) - Y_i(Z_i, 0)}{D_i = 0}
        \end{aligned} \right) \\
        &+ \left( \frac{1 - \Prob{D_i(1) = 1, D_i(0) = 0} }{
            \Prob{D_i(1) = 1, D_i(0) = 0}} \right)
        \left( \begin{aligned}
            &\Egiven{Y_i(Z_i, 1) - Y_i(Z_i, 0)}{D_i(1) = 0 \text{ or } D_i(0)=1} \\ 
            &  - \E{Y_i(Z_i, 1) - Y_i(Z_i, 0)}
        \end{aligned} \right)
    \end{aligned} \right]
\end{align*}

The selection bias terms come from systematic differences between the treated and untreated groups, differences not fully unexplained by $\vec X_i$.
The group differences represent the fact that a matching estimator gives an average effect on the treated group and, when selection-on-observables does not hold, this is systematically different from the average effect \citep{heckman1998characterizing}.\footnote{
    The selection-on-observables approach could, instead, focus on the average effect on treated populations (as do \citealt{keele2015identifying}).
    This runs into a problem of comparisons: CM estimates would give average effects on different treated groups.
    The CM estimate for the ADE on treated gives the ADE local to the $Z_i = 1$ treated group, while the AIE estimate gives the AIE local to the $D_i = 1$ group.
    In this way, these ADE and AIE on treated terms are not comparable to each other, so I focus on the true average terms.
}
The group differences term is longer for the average indirect effect estimate, because the indirect effect is comprised from the effect of $D_i$ local to $Z_i$ compliers; a matching estimator gets the average effect on treated, and the longer term adjusts for differences with the complier average effect.
