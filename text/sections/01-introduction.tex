%\section{Introduction}
%\label{sec:intro}
% \textbf{The introduction formula \url{https://blogs.ubc.ca/khead/research/research-advice/formula}.}
%\textbf{Hook:}
Economists use natural experiments to credibly answer social questions, when an experiment was infeasible.
For example, does winning access to health insurance causally improve health and well-being \citep{finkelstein2008oregon}?
Natural experiments are settings which answer these questions, but give little indication of how these effects came about.
Causal Mediation (CM) aims to estimate the mechanisms behind causal effects, by estimating how much of the treatment effect operates through a proposed mediator.
For example, do causal gains from winning access to health insurance come mostly from using healthcare more often, or are there other direct effects?
This study of mechanisms behind causal effects broadens the economic understanding of social settings studied with natural experiments.
This paper shows that the conventional approach to estimating CM effects is inappropriate in a natural experiment setting, provides a theoretical framework for how bias operates, and develops an approach to correctly estimate CM effects under alternative assumptions.
These methods contrast the current practice in applied economics of providing suggestive evidence of mechanisms, which neither identifies nor quantifies mechanisms behind causal effects.

% Paragraph here: how do applied economists currently do it?
% Summarise my work with the Finkelstein+ data.

%\textbf{Question:}
This paper starts by considering conventional CM methods in a natural experiment setting.
Conventional CM methods rely on assuming the initial treatment, and the subsequent mediator, are both ignorable \citep{imai2010identification}.
Assuming the mediator is as-good-as-randomly assigned conveniently ignores selection by assuming either (1) people na\"ively made decisions to take or refuse a mediator, or (2) a researcher controlled for everything relevant to this decision.
This assumption might be reasonable when studying single-celled organisms in a laboratory --- their ``decisions'' are simple and mechanical.
Social scientists, however, study humans who make complex choices based on costs, benefits, and preferences --- which are only partially observed by researchers, at best.
For example, winners of the Oregon Health Insurance Experiment wait-list were randomly chosen by a lottery, but went on to chose to visit healthcare providers of their own free will.
In practice, the main setting where mediator ignorability becomes credible is when researchers find another natural experiment affecting the mediator --- a rare occurrence given how difficult it is to find one source of random variation for a treatment, let alone another independent source for a mediator, at the same time.

%\textbf{Antecedents:}
The applied economics literature has been hesitant to use explicit CM methods, instead providing suggestive evidence for mechanisms,\footnote{
    See \cite{blackwell2024assumption} for an overview of this approach, from the empirical politics literature.
} sometimes accompanied by a practice of controlling for a proposed mediator.
Neither of these approaches have a causal interpretation.
A new strand of the econometric literature has developed estimators for explicit CM analyses under a variety of strategies.
These include overlapping quasi-experimental research designs \citep{deuchert2019direct,frolich2017direct}, functional form restrictions \citep{heckman2015econometric,heckman2013understanding}, partial identification \citep{flores2009identification}, or a hypothesis test of full mediation through observed channels \citep{kwon2024testing} --- see \cite{huber2019review} for an overview.\footnote{
    An alternative method to estimate CM effects is ensuring treatment and mediator ignorability holds by a running two randomised controlled trials for both treatment and mediator, at the same time.
    This set-up has been considered in the literature previously, in theory \citep{imai2013experimental} and in practice \citep{ludwig2011mechanism}.
}
The new literature has arisen in implicit acknowledgement that suggestive evidence of mechanisms, or a conventional approach to CM, can lead to biased inference and needs alternative methods for credible inference.

%\textbf{Value-added:}
I develop a framework showing exactly how selection bias contaminates conventional CM estimates when mediator choices are driven by unobserved gains --- settings where none of the natural experiment research designs in the previously cited papers apply (i.e., the mediator is not ignorable).
This provides a rigorous warning to applied economists against uncritically applying conventional CM methods to investigate mechanisms --- as is common in some applied fields of epidemiology, psychology, and medicine.

Selection based on costs and benefits is at odds with assuming a mediator is as-good-as randomly assigned in an observational setting, so I import methods grounded in labour economic theory to solve the identification problem.
This approach identifies CM effects with a control function, via the marginal effect of the mediator, and requires three main assumptions.
(1) Mediator take-up must respond only positively to the initial treatment (monotonicity), which implies mediator selection follows a selection model.
(2) Mediator take-up is motivated by mediator benefits.
(3) A valid instrument for mediator take-up must exist, to avoid relying on parametric assumptions on unobserved selection.
While these assumptions are strong, they are plausible in many applied settings.
Mediator monotonicity aligns with conventional theories for selection-into-treatment, and is accepted widely in many applications using an instrumental variables research design.
Selection based on costs and benefits is central to economic theory, and is the dominant concern for judging observational designs that identify causal effects.
Access to valid instrumental variation is a strong condition, though is important to avoid further modelling assumptions; the most compelling example is using variation in mediator take-up costs as an instrument.

Applying the new methods to the Oregon Health Insurance Experiment shows that unobserved selection matters in an analysis of a real-world natural experiment.
A substantial portion of the wait-list lottery's impact on self-reported health and happiness is mediated indirectly through extra healthcare usage, after instrumenting for healthcare usage with respondents' usual provider.
A conventional CM analysis would put this indirect mediated share at practically zero, so that my methods expose that negative selection into healthcare usage would be hiding evidence for this mechanism.
These estimates replace claims on mechanisms with credible causal evidence that extra healthcare use mediates a sizeable share of the Medicaid-lottery's benefits, avoiding claims based only on suggestive conjecture.

The methods I propose for CM analyses are not perfect for every setting: the structural assumptions are strong, and are tailored to selection-into-mediator based on the economic principle of selection based on costs and benefits.
Indeed, this approach provides no safe harbour for estimating CM effects if these structural assumptions do not hold true.
This approach imports insights from the instrumental variables literature, connecting the influential \cite{imai2010identification} approach to CM with the economics literature on selection-into-treatment and marginal treatment effects \citep{vytlacil2002independence,heckman2004using,heckman2005structural,florens2008identification,kline2019heckits}.
%\footnote{
%    Indeed, this paper does not invent control function methods, instead noting their applicability in this setting.
%    See \cite{wooldridge2015control,imbens2007nonadditive} for general overviews of the approach.
%}
\cite{frolich2017direct} have previously explored identification of CM effects with a control function in the context of two instruments (one each for treatment and mediator) and a continuous mediator.
This paper considers CM effects via the marginal effect of a binary mediator, with a different identification analysis and estimation strategies.

%\textbf{Road-map:}
This paper proceeds as follows.
\autoref{sec:lottery} describes the dominant approach in economics for studying mechanisms behind treatment effects, illustrating with data from the Oregon Health Insurance Experiment.
% and surveys economic research.
\autoref{sec:mediation} introduces the formal framework for CM, and develops expressions for bias in CM estimates in natural experiments.
\autoref{sec:applied} describes this bias in applied settings with (1) a regression framework, (2) a setting with selection based on costs and benefits.
\autoref{sec:selectionmodel} purges bias from CM estimates by identifying CM effects with a control function adjustment, via the marginal effect of the mediator.
\autoref{sec:controlfun} demonstrates how to estimate CM effects with this approach, with either parametric or semi-parametric methods, and gives simulation evidence.
\autoref{sec:oregon} returns to the Oregon Health Insurance Experiment, providing credible estimates of effects on self-reported health and well-being mediated through healthcare usage.
\autoref{sec:conclusion} concludes.
