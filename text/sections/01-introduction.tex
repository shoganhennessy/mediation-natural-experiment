%\section{Introduction}
%\label{sec:intro}
Conventional CM methods rely on a selection-on-observables assumption, which may not hold true in observational work.
I explicitly connect the assumptions behind CM methods to those of selection into treatment in classic labour and observation economic research \citep{heckman2005structural}.
When a mediator, here education, is not randomly assigned then conventional CM methods for estimating direct and indirect effects are contaminated by selection bias.
I write this as both a non-parametric non-identification result, and with a model-based regression framework with a correlated error term(e.g., as in the \citealt{imai2010identification} linear model approach).
Structural assumptions could solve the identification problem, for example if selection into education follows a Roy model or errors terms have a known distribution \citep{heckman1979sample}.


Intro paragraph on the selection bias term.
These are similar, in spirit, to selection bias of \cite{heckman1998characterizing}.
Also give a result closely related to bad control bias (also known as M-bias, Ding 2015, Cinelli 2022).
Also similar to the non-identification result of Bugni Canay McBride (2024).





This paper proceeds as follows.
\autoref{sec:mediation} introduces causal mediation, and develops selection bias in mediation estimates in natural experiments.
\autoref{sec:selection} describes the selection bias in applied settings, a regression framework and a natural experiment with selection based on costs and benefits.
\autoref{sec:controlfun} identifies and estimates average direct and indirect effects under a selection model, with simulation evidence that this approach purges bias in mediation estimates.
\autoref{sec:conclusion} concludes.
% \textbf{Re-write with this, \url{https://blogs.ubc.ca/khead/research/research-advice/formula}.}
