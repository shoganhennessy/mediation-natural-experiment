%\section{Introduction}
%\label{sec:intro}
% \textbf{The introduction formula \url{https://blogs.ubc.ca/khead/research/research-advice/formula}.}
%\textbf{Hook:}
Economists use natural experiments to credibly answer social questions, without the trouble of guiding randomisation of what they study.
Did Vietnam-era military service lead to income losses?
Does access to health insurance lead to employment gains?
Do transfer payment  lead to measurable long-run economic gains?
Quasi-experimental variation gives methods to answer these questions, but give no indication of how these effects came about.
Causal Mediation (CM) aims to estimate the mechanisms behind causal treatment effects, by estimating how much of the treatment effect operates through a proposed mediator.
For example, how much of the (causal) gain from a transfer payment came from individuals choosing to attend higher education?
This paper shows that the conventional approach to estimating CM effects is inappropriate in a natural experiment setting, giving a theoretical framework for how large bias terms are in the real world, and an approach to correctly estimate CM effects under minimal structural assumptions.

%\textbf{Question:}
This paper starts by answering the following question: what does a selection-on-observables CM approach actually estimate when the mediator is not ignorable?
Estimates for the average direct and indirect effects are contaminated by bias terms --- a sum of selection bias and non-parametric group differences.
I then show how this bias operates in an applied regression framework, with bias coming from a correlated error term, showing that the bias term grows larger with the degree of unexplained selection.
If individuals have been choosing whether to partake in a mediator based on expected costs and benefits (i.e., following a rational maximisation process), then assuming the mediator is ignorable gives unlikely implications for choice behaviour.
% I develop a testable assumption, that $R^2_u = 1$, in this framework, which is likely to be rejected in the majority of applied examples.
This means the identifying assumption for conventional CM methods are unlikely to hold, and likely lead to biased inference in natural experiment settings.

I consider an alternative control function approach to estimating mediation effects.
This approach solves the identification problem by instead placing a structural assumption for selection into the mediator (monotonicity), and assumes the researcher has a valid instrument for mediator take-up.
These assumptions may hold in real-world natural experiment settings.
% In an accompanying applied paper, \cite{hogan2025direct}
Mediator monotonicity is in-line with conventional theories for selection-into-treatment, and is accepted widely in many applications using an instrumental variables research design.
The existence of a valid instrument is a stronger assumption, which will not hold in every applied example, though is important to avoid parametric assumptions. %on unexplained mediator selection.
The most compelling example is using data on the cost of mediator take-up as a first-stage instrument, if it varies between individuals for exogenous reasons and is strong in explaining compliance.
Using an instrument avoids parametric assumptions on unexplained mediator selection, though limits the wider applicability of the method.
This approach is not perfect: it provides no harbour for estimating CM effects if these structural assumptions do not hold true, though performs no worse than conventional CM methods in this case.

%\textbf{Antecedents:}
The most popular approach to CM estimates direct and indirect effects by assuming that a treatment is ignorable, and then assuming that a mediator is ignorable conditional on the treatment assignment \citep{imai2010identification}.
This approach arose in the statistics literature, and is widely used in epidemiology, medicine, and psychology to estimate mediation effects in observational studies.
The applied economics literature has not picked up this practice, partially in an understanding that these assumptions are invalid in most observational settings.
Indeed, a new strand of the econometric literature has developed estimators for CM effects under overlapping quasi-experimental research designs \citep{deuchert2019direct,frolich2017direct}, a partial identification approach \citep{flores2009identification}, or testing full mediation through observed channels \citep{kwon2024testing} --- see \cite{huber2019review} for an overview.
The new literature has arisen in partial acknowledgement that a conventional selection-on-observables approach to CM in an applied setting can lead to biased inference, and needs alternative methods for credible inference in many cases.
This paper makes this part explicit, showing exactly how a conventional approach to CM in a natural experiment can fail in practice.

%\textbf{Value-added:}
This paper considers the case when it is not credible to assume the mediator is ignorable (e.g., none of the research designs above apply), leveraging classic labour economic theory for selection-into-treatment to identify direct and indirect effects.
A selection-on-observables approach to CM in this setting suffers from bias of the same flavour as classic selection bias \citep{heckman1998characterizing}, plus additional bias from group differences.
The group differences-bias is a non-parametric version of bad controls bias, which has only previously been studied in a linear setting \citep{cinelli2024crash,ding2015adjust}.

Throughout, I use the \cite{roy1951some} model as a benchmark for judging the \cite{imai2010identification} mediator ignorability assumption in a natural experiment setting, and find it unlikely to hold in practice.\footnote{
    An alternative method to estimate CM effects is ensuring sequential ignorability holds by a running randomised controlled trial for both treatment and mediator at the same time.
    This setting has been considered in the literature previously, in theory \citep{imai2013experimental} and in practice \citep{ludwig2011mechanism}.
}
This motivates a solution to the identification problem inspired by classic labour economic work, which also uses the Roy model as a benchmark \citep{heckman1979sample,heckman1990empirical}.
I follow the lead of these papers by using a control function approach to correct for the bias developed above.
This approach assumes mediator monotonicity, to ensure the mediator follows a selection model \citep{vytlacil2002independence}, and a valid instrument for mediator take-up, to avoid parametric assumptions on unobserved selection \citep{heckman2004using}.
Doing so is as an extension of using instruments to identify CM effects --- already noted by \citep{frolich2017direct}.\footnote{
    Indeed, this paper does not improve on control function methods in any way, instead noting its applicability in this setting.
    See \cite{frolich2017direct} for the newest development of control function methods with instruments, and \cite{imbens2007nonadditive} for a general overview of the approach.
}
Using a control function to estimate CM effects builds on the influential \cite{imai2010identification} approach, marrying the CM literature with labour economic theory on selection-into-treatment for the first time. 

This paper proceeds as follows.
\autoref{sec:mediation} introduces CM, and develops expressions for the bias in mediation estimates in natural experiments.
\autoref{sec:selection} describes this bias in applied settings with (1) a regression framework, (2) a setting with selection based on costs and benefits.
%, (3) a short survey of empirical practice.
\autoref{sec:controlfun} solves the identification problem with a control function, assuming a mediator follows a selection model and a researcher observes exogenous variation in cost of mediator take-up, and gives simulation evidence.
\autoref{sec:conclusion} concludes.
