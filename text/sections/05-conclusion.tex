%%%%%%%%%%%%%%%%%%%%%%%%%%%%%%%%%%%%%%%%%
%% Conclusion section
\section{Summary and Concluding Remarks}
\label{sec:conclusion}

This paper studies the returns to higher education, using IV methods from the epidemiology literature and adjustments from the causal mediation literature to tackle violations of the exclusion restriction.
First, I derive identification of the average mechanism effect under a selection-on-observables type assumption, and partial identification when unobserved selection confounding.
I apply these methods to a sample of retirement age Americans in the years 1990--2021, using genetic information to instrument for higher education, estimating that higher education leads to roughly 40\% higher earnings (point estimates), or between 8--44\% higher earnings (partial bounds).
Additionally, women had significantly higher returns to higher education over this time period.

The methods here provide alternatives to assuming the exclusion restriction in empirical applications of IV models, so can be useful in sensitivity analyses for any application of IV methods.
Mendelian randomisation is a particularly useful application of IV methods, though the exclusion restriction is particularly problematic in practice.
The approach allows researchers to use MR to study effects of both health conditions and behaviours with significant selection-into-treatment concerns, such as higher education.
