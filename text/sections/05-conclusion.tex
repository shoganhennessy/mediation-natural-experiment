%%%%%%%%%%%%%%%%%%%%%%%%%%%%%%%%%%%%%%%%%
%% Conclusion section
\section{Summary and Concluding Remarks}
\label{sec:conclusion}

This paper has studied a selection-on-observables approach to CM in a natural experiment setting.
I have shown the pitfalls of using the most popular methods for estimating direct and indirect effects without a clear case for the mediator being ignorable.
Using the Roy model as a benchmark, a mediator is unlikely to be ignorable in natural experiment settings, and the bias terms likely crowd out inference regarding CM effects.

This paper has contributed to the growing CM literature in economics, integrating labour economic theory for how individuals select into as a way of judging CM methods.
It has drawn on the classic literature, and pointed to already-in-use control function methods as a compelling way of estimating direct and indirect effects in a natural experiment.
Further research could build on this approach by suggesting efficiency improvements, adjustments for common statistical irregularities (say, cluster dependence), or integrating the control function as an additional robustness in the growing double robustness literature \citep{huber2019review,bia2024double}.

This paper has not lit the way for applied researchers to use CM methods unabashedly, with or without a control function adjustment.
The structural assumptions are strong and large sample sizes are needed; if the assumptions are broken, then the control function method does not improve on a na\"ive selection-on-observables approach to CM estimates.
And yet, there are likely settings in which the structural assumptions are credible.
Mediator monotonicity aligns well with economic theory in many cases, and it is plausible for researchers to study big data settings with exogenous variation in mediator take-up costs.
In these cases, this paper opens the door to identifying mechanisms behind treatment effects in natural experiment settings.
% The approach could be used in AB tests, where a firm randomises a treatment and costs of a suspected mediator (if they do not want to also randomise a mediator fully).
