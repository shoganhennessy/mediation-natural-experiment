\noindent
Natural experiments are a cornerstone of applied economics, providing settings for estimating causal effects with a compelling argument for treatment randomisation.
Applied researchers often investigate mechanisms behind treatment effects by controlling for a mediator of interest, alluding to Causal Mediation (CM) methods for estimating direct and indirect effects (CM effects).
This approach to investigating mechanisms unintentionally assumes the mediator is quasi-randomly assigned --- in addition to quasi-random assignment of the initial treatment.
Individuals' choice to take (or refuse) a mediator based on costs and benefits is inconsistent with this assumption, suggesting in-practice estimates of CM effects are contaminated by bias terms which crowd out inference.
I consider an alternative approach to credibly estimate CM effects, using control function methods and relying on instrumental variation in mediator take-up cost.
Simulations confirm this approach corrects for bias in conventional CM estimates, providing parametric and semi-parametric methods.
This approach gives applied researchers an alternative method to estimate CM effects when an initial treatment is quasi-randomly assigned, but the mediator is not, as is common in natural experiments.
% I illustrate the approach by estimating the proportion of the causal effect of genes associated with education that operates via higher education decisions versus a direct genetic channel.
% Finally, I provide an implementation of this method in the \textit{R} package \textit{mediate-selection}, offering an accessible tool to investigate mechanisms in natural experiment settings.

\vspace{0.5cm}
\noindent
\textbf{Keywords:}
Direct/indirect effects, quasi-experiment, selection, control function.

\vspace{0.1cm}
\noindent
\textbf{JEL Codes:}
C21, C31.
