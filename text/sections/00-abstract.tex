\noindent
Natural experiments are a cornerstone of applied economics, providing settings for estimating causal effects with a compelling argument for treatment ignorability.
Applied researchers often investigate mechanisms behind treatment effects by controlling for a mediator of interest, alluding to Causal Mediation (CM) methods for estimating direct and indirect effects (CM effects).
This approach to investigating mechanisms unintentionally assumes the mediator is quasi-randomly assigned --- in addition to quasi-random assignment of the initial treatment.
Individuals' choice to take (or refuse) a mediator based on costs and benefits is inconsistent with this assumption, suggesting in-practice estimates of CM effects are biased in natural experiment settings.
I solve for these explicit bias terms, which imitate classical selection bias for average causal effects and crowd out inference on direct and indirect effects, and consider an alternative approach to credibly estimate CM effects.
The approach uses a control function adjustment, pertinent for when selection-into-mediator is driven by unobserved costs and benefits, and relies on instrumental variation in mediator take-up cost.
Simulations confirm that this method corrects for selection bias in conventional CM estimates, providing both parametric and semi-parametric methods.
% I illustrate the approach by estimating the proportion of the causal effect of genes associated with education that operates via higher education decisions versus a direct genetic channel.
% Finally, I provide an implementation of this method in the \textit{R} package \textit{mediate-selection}, offering an accessible tool to investigate mechanisms in natural experiment settings.
This approach gives applied researchers an alternative method to estimate CM effects when they can only establish a credible argument for quasi-random assignment of the initial treatment, and not a mediator, as is common in natural experiments.

\vspace{0.5cm}
\noindent
\textbf{Keywords:}
Direct/indirect effects, quasi-experiment, selection, control function.

\vspace{0.1cm}
\noindent
\textbf{JEL Codes:}
C21, C31.
