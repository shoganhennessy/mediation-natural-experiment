\noindent
Natural experiments are a cornerstone of applied economics, providing settings for estimating causal effects with a compelling argument for treatment randomisation.
Applied researchers often investigate mechanisms with a suggestive investigation of mechanisms, giving shaky evidence for mechanisms behind causal effects.
Causal Mediation (CM) provides an alternative, robust framework for identifying and estimating direct and indirect effects (CM effects) for causal mechanisms, but relies on assumptions which are implausible in natural experiment settings.
In particular, conventional CM estimates rely on assuming the mediator is as-good-as-randomly assigned.
When individuals select into a mediator based on costs and benefits, this assumption fails, undermining causal inference.
I develop an alternative approach to credibly estimate CM effects, using control function methods and instrumental variation in take-up costs to avoid unrealistic assumptions.
Simulations confirm this approach corrects for bias in conventional CM estimates, providing parametric and semi-parametric methods.
%I illustrate the approach by analysing the effect of health insurance through healthcare in the Oregon Health Insurance Experiment.
% Finally, I provide an implementation of this method in the \textit{R} package \textit{mediate-selection}, offering an accessible tool to investigate mechanisms in natural experiment settings.
This approach gives applied researchers an alternative method to estimate CM effects when an initial treatment is quasi-randomly assigned, but the mediator is not, as is common in natural experiments.

\vspace{0.5cm}
\noindent
\textbf{Keywords:}
Direct/indirect effects, quasi-experiment, selection, control function, MTEs.

\vspace{0.1cm}
\noindent
\textbf{JEL Codes:}
C21, C31.
