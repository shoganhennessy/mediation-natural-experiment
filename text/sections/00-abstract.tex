\noindent
Natural experiments are a cornerstone of applied economics, providing settings for estimating causal effects of a treatment with a compelling argument for treatment ignorability.
Economists are often interested in understanding the mechanisms through which causal effects operate, and mediation methods aim to estimate these components.
However, conventional mediation methods rely on a selection-on-observables assumption, assuming the mediator is conditionally ignorable --- in addition to the natural experiment for the original treatment.
This paper shows that conventional estimates of mediation effects are contaminated by selection bias when the mediator is not ignorable.
Using the case of a Roy model for a mediator, I show that individuals' selection based on expected gains and costs is inconsistent with mediator ignorability without implausible behavioural assumptions.
I develop a control function approach, which correctly estimates mediation effects when selection into the mediator follows a selection model, using cost of mediator take-up as an instrument.
Simulations confirm that this method corrects for selection bias in conventional mediation estimates, and performs comparably to a selection-on-observables approach when the mediator selection does not follow a selection model.
I illustrate the approach by estimating the proportion of the causal effect of genes associated with education that operates via a direct genetic channel versus indirectly through extended schooling.
Finally, I provide an implementation of this method in the \textit{R} package \textit{mediate-controlfun}, offering an accessible tool for robust mediation analysis in natural experiment settings.

\vspace{0.5cm}
\noindent
\textbf{Keywords:}
Causal Mediation.

\vspace{0.1cm}
\noindent
\textbf{JEL Codes:}
D31, D91, I24, J24, Z00.
