\noindent
Natural experiments are a cornerstone of applied economics, providing settings for estimating causal effects with a compelling argument for treatment randomisation, but give little indication of the mechanisms behind causal effects.
Causal Mediation (CM) is a framework for sufficiently identifying a mechanism behind the treatment effect, decomposing it into an indirect effect channel through a mediator mechanism and a remaining direct effect.
By contrast, a suggestive analysis of mechanisms gives necessary but not sufficient evidence.
Conventional CM methods require that the relevant mediator mechanism is as-good-as-randomly assigned; when people choose the mediator based on costs and benefits (whether to visit a doctor, to attend university, etc.), this assumption fails and conventional CM analyses are at risk of bias.
I propose an alternative strategy that delivers unbiased estimates of CM effects despite unobserved selection, using instrumental variation in mediator take-up costs.
The method identifies CM effects via the marginal effect of the mediator, with parametric or semi-parametric estimation that is simple to implement in two stages.
Applying these methods to the Oregon Health Insurance Experiment reveals a substantial portion of the Medicaid lottery's effect on subjective health and well-being flows through increased healthcare usage --- an effect that a conventional CM analysis would mistake.
This approach gives applied researchers an alternative method to estimate CM effects when an initial treatment is quasi-randomly assigned, but a mediator mechanism is not, as is common in natural experiments.

\vspace{0.5cm}
\noindent
\textbf{Keywords:}
Direct/indirect effects, quasi-experiment, selection, MTEs.

\vspace{0.1cm}
\noindent
\textbf{JEL Codes:}
C21, C31.
