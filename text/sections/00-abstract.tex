\noindent
Natural experiments are a cornerstone of applied economics, providing settings for estimating causal effects with a compelling argument for treatment randomisation.
Common practice investigates the mechanisms with suggestive evidence, giving at best shaky evidence.
Causal Mediation (CM) provides an alternative framework for identifying and estimating direct and indirect effects (CM effects), yet conventional methods assume the mediator is as-good-as-randomly assigned.
When people choose the mediator based on costs and benefits (whether to visit a doctor, to attend university, etc.), this assumption fails and conventional CM estimates are biased.
I propose a control function strategy that uses instrumental variation in mediator take-up costs, delivering unbiased direct and indirect effects when selection is driven by unobserved gains.
The method works with either parametric or semi-parametric methods, and is simple to implement in two stages.
Applying these methods to the Oregon Health Insurance Experiment reveals a substantial portion of the wait-list lottery's effect on self-reported health and happiness flows through increased healthcare usage --- an effect that a conventional CM analysis would mistake.
This approach gives applied researchers an alternative method to estimate CM effects when an initial treatment is quasi-randomly assigned, but the mediator is not, as is common in natural experiments.

\vspace{0.5cm}
\noindent
\textbf{Keywords:}
Direct/indirect effects, quasi-experiment, selection, control function, MTEs.

\vspace{0.1cm}
\noindent
\textbf{JEL Codes:}
C21, C31.
