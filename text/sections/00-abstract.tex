\noindent
Natural experiments are a cornerstone of applied economics, providing settings for estimating causal effects with a compelling argument for treatment ignorability.
Economists are often interested in understanding the mechanisms through which causal treatment effects operate, and Causal Mediation (CM) methods aid this by estimating how much of the treatment effect operates through a proposed mediator.
The most popular CM approach relies on assumptions which are unrealistic in natural experiment settings: assuming the mediator is conditionally ignorable --- in addition to the ignorability argument for the initial treatment.
This paper shows that this approach leads to biased inference, solving for explicit bias terms when the mediator is not ignorable.
Using the case of a Roy model for a mediator, I show that individuals' selection based on expected gains and costs is inconsistent with mediator ignorability without implausible behavioural assumptions, and that bias terms are large in practice.
I show a control function approach, which overcomes these hurdles if monotonicity holds, using cost of mediator take-up as an instrument.
Simulations confirm that this method corrects for persistent bias in conventional CM estimates, and performs comparably to a selection-on-observables approach when the structural assumptions do not hold.
% I illustrate the approach by estimating the proportion of the causal effect of genes associated with education that operates via a direct genetic channel versus indirectly through extended schooling.
% Finally, I provide an implementation of this method in the \textit{R} package \textit{mediate-controlfun}, offering an accessible tool for robust mediation analysis in natural experiment settings.
This approach gives applied researchers a practical method to estimate CM effects when they can only establish a credible argument for randomisation of the initial treatment, as is common in natural experiments.

\vspace{0.5cm}
\noindent
\textbf{Keywords:}
Direct/indirect effects, quasi-experiment, selection, control function.

\vspace{0.1cm}
\noindent
\textbf{JEL Codes:}
D31, D91, I24, J24, Z00.
