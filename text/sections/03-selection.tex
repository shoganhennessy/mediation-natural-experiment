\section{Causal Mediation in Applied Settings}
\label{sec:selection}
In this section, I further develop the issue of selection in causal mediation estimates. First, I show the non-parametric bias terms from above can be written as omitted variables bias in a regression framework.
Second, I show how selection bias operates in an applied model for selection into a mediator based on costs and benefits.

\subsection{Regression Framework}
\label{sec:regression}
Inference for direct and direct effects can be written in a regression framework, showing how correlation between the error term and the mediator persistently biases estimates.
Write $Y_i(Z, D)$ as a sum of observed factors $Z_i, \vec X_i$ and unobserved factors, $U_{1,i}, U_{0,i}$ (following the notation of \citealt{heckman2005structural}).
Put $\mu_D(Z; \vec X_i) = \Egiven{Y_i(Z_i, 0)}{\vec X}$, to give a representation of the average direct and indirect effects.
\begin{align*}
    \E{Y_i(Z_i, D_i(1)) - Y_i(Z_i, D_i(0))}
    &= \mathbb{E} \Big[ \Big( D_i(1) - D_i(0) \Big)
        \times \Big( \mu_1(Z_i; \vec X_i) - \mu_0(Z_i; \vec X_i) \Big) \Big], \\
    \E{Y_i(1, D_i(Z_i)) - Y_i(0, D_i(Z_i))}
        &= \mathbb{E} \Big[ \mu_{D_i}(1; \vec X_i) - \mu_{D_i}(0; \vec X_i) \Big].
\end{align*}
Then define the error terms.
\[ U_{0,i} = Y_i(Z_i, 0) - \mu_0(Z_i; \vec X_i),\;\;\;\;
U_{1,i} = Y_i(Z_i, 1) - \mu_1(Z_i; \vec X_i) \]
With this notation, observed data $Z_i, D_i, Y_i$ take the following representation, which characterises direct effects, indirect effects, and bias from selection.
\begin{align}
    \label{eqn:parametric-firststage}
    D_i &= \phi + \pi Z_i + \varphi(\vec X_i) + \eta_i  \\
    \label{eqn:parametric-secondstage}
    Y_i &= \alpha + \beta D_i + \gamma Z_i + \delta Z_i D_i
    + \zeta(\vec X_i)
    + \underbrace{U_{0,i} + D_i \left( U_{1,i} - U_{0,i} \right)}_{
        \text{Correlated error term.}}
\end{align}
First-stage \eqref{eqn:parametric-firststage} is identified.
$\alpha, \zeta(\vec X_i)$ are the intercept terms, and $\beta, \gamma, \delta$ are values that comprise mediation effects --- all whose values may depend on $\vec X_i$, see \autoref{appendix:regression-model} for full definitions.
The average direct and indirect effects are averages of these terms, across the distribution of $\vec X_i$.
\begin{align*}
    \E{Y_i(Z_i, D_i(1)) - Y_i(Z_i, D_i(0))}
        &= \E{\pi \left( \beta +  Z_i \delta \right)}, \\
    \E{Y_i(1, D_i(Z_i)) - Y_i(0, D_i(Z_i))}
        &= \E{\gamma + \delta D_i}.
\end{align*}
By construction, $U_i = U_{1, i} - U_{0, i}$ is an unobserved confounder.
The regression estimates of \autoref{eqn:parametric-secondstage} give unbiased estimates only if $D_i$ is also conditionally ignorable: $D_i \indep  U_i$.
If not, then regression estimates suffer from omitted variables bias if they do not adjust for the unobserved confounder $U_i$.

\subsection{Selection on Costs and Benefits}
The key to noting that CM is at risk of bias is noting that $D_i \indep  U_i$ is unlikely to hold in applied settings.
Without an identification strategy for $D_i$, in addition to that for $Z_i$, bias will persist, given how we conventionally think of selection into treatment.

Consider a model where individual $i$ selects into a mediator after $Z_i, \vec X_i$, based on costs and benefits.
The model has $i$ taking the mediator if the benefits of doing so exceed the costs.
Write $C_i$ for individual $i$'s costs of taking mediator $D_i$, and $\indicator{.}$ for the indicator function.
\[ D_i \left( z' \right) = \indicator{
    \underbrace{Y_i\left( z', 1 \right) - Y_i\left( z', 0 \right)}_{\text{Benefits}}
    \geq \underbrace{C_i}_{\text{Costs}}}, \;\;\; \text{for } z'=0,1\]
Paragraph here talking about why the Roy model is useful.
\citep{roy1951some,heckman1990empirical}.

Write in here with the updated notation of the above, and lead to the $D_i \not\indep U_i$ statement, unless $\vec X_i$ is incredibly rich.


\[ C_i(Z_i) = \mu_{C}(Z_i; \vec X_i) + U_{C,i} \]

And so we can write the first-stage selection equation in full.
\begin{align*}
    D_i(z) &= \indicator{
        \underbrace{U_{C, i} + U_{0, i} - U_{1, i}}_{\text{Unobserved}}
        \leq
        \underbrace{
            \mu_1(z; \vec X_i) - \mu_0(z; \vec X_i)- \mu_C(z; \vec X_i)}_{\text{Observed}}}
\end{align*}



\subsection{Applied Settings}

Three parapgraphs on what goes on in empirical settings.
Survey the papers, and speak about it heavily in one paragraph.

table:

name | $Z \to Y$ | design for $Z$ | Primary mediatory | controls | Possible $U$.






\newpage
\subsection{Older Writing}



This section connects causal mediation, without assuming the mediator is randomly assigned (i.e., under selection), with classic labour economics models for selection into treatment.

\subsection{Selection Model Representation}
\label{sec:selection-model}
The IV literature assumes a first-stage monotonicity condition, where randomised $Z_i$ influences mediator $D_i$ in at most one direction.
\begin{definition}
    \label{dfn:firststage-monotonicity}
    First-stage Monotonicity \citep{imbens1994identification}.
    \begin{equation}
        \label{eqn:firststage-monotonicity}
        \Prob{ D_i(1) \geq D_i(0)} = 1    
    \end{equation}
\end{definition}

Assuming \ref{dfn:firststage-monotonicity}\eqref{eqn:firststage-monotonicity}
in a mediation setting opens mediation to the wide literature on IV and selection models for identification in the presence of selection. 
\begin{theorem}
    \label{thm:selection-model}
    Under monotonicity, mediator $D_i$ can be represented by a selection model. \\
    Suppose \ref{dfn:firststage-monotonicity}\eqref{eqn:firststage-monotonicity} holds, then there is a function $\mu(.)$ and random variable $U_i$ such that $D_i$ takes the following form.
    \[ D_i(z) = \indicator{ \mu(z) \geq U_i}, \;\; \forall z = 0,1 \]
\end{theorem}
\begin{proof}
    Special case of the \cite{vytlacil2002independence} equivalence result; see \autoref{appendix:selection-model}.
\end{proof}

\autoref{thm:selection-model} is a powerful result: it says that at the cost of assuming monotonicity (as is done in the IV literature), then selection into $D_i$ takes a latent index form, and opens up identification in a mediation context to the wide literature on identifying treatment effects in selection models.

\subsection{A Regression Framework for Direct and Indirect Effects}

\subsection{Selection into Education}
In the education context, point identifying direct and indirect effects requires the \textit{researcher controls for all sources of selection-into-education}.

While this assumption may hold true in two-way randomised experiments (e.g., in a laboratory or two-way RCT), it is unlikely to hold in the case of quasi-experimental variation in $Z$, or when modelling education as a mediator --- absent a separate identification strategy for education $D$.
To expand this point in an econometric selection-into-treatment framework, suppose selection follows a Roy model, where individual $i$ weighs the costs and benefits of completing education.
\[ D_i(Z_i) = \indicator{
    \underbrace{C_i(Z_i)}_{\text{Costs}}
    \leq
    \underbrace{Y_i(Z_i, 1) - Y_i(Z_i, 0)}_{\text{Gains}}} \]
Education choice $D_i(z)$ is clearly related to $Y_i(z, d)$ in this model, so let's see what the equation looks like in terms of sequential ignorability.
As above, decompose costs into observed and unobserved factors.
\[ C_i(Z_i) = \mu_{C}(Z_i; \vec X_i) + U_{C,i} \]

And so we can write the first-stage selection equation in full.
\begin{align*}
    D_i(z) &= \indicator{
        \underbrace{U_{C, i} + U_{0, i} - U_{1, i}}_{\text{Unobserved}}
        \leq
        \underbrace{
            \mu_1(z; \vec X_i) - \mu_0(z; \vec X_i)- \mu_C(z; \vec X_i)}_{\text{Observed}}}
\end{align*}
Sequential ignorability, where $Y_i(z, d) \indep D_i(z') \; | \; \vec X_i$, would then require that $\Egiven{U_{0, i} - U_{1, i}}{D_i} = 0$.
In the Roy model above, this would assume every single contribution for returns to education is contained in $\vec X_i$; if there are any unobserved sources by which people have systematically different returns to education, then they would select into education based on this, and bias na\"ive mediation estimates.
This is unlikely to hold true, unless there is another identification strategy for $D_i$ --- in addition to the one used for $Z_i$.

\subsection{Estimating Direct and Indirect Effects}

Quasi-experimental work does not take the assumption of ``selection-on-observables'' at face value without an explicit research design \citep{angrist2009mostly} or modelling approach to address this issue.

A classical approach to modelling this issue, is a selection model approach \citep{heckman1974shadow,heckman1979sample}.
The approach assume $U_0, U_1$ follow a known distribution (e.g, bivariate normal), and estimates the regression via maximum likelihood.
Alternatively, a control function approach estimates the system in two stages, avoiding (some) distributional assumptions if an instrument is used, at the cost of efficiency.
In the following, I estimate direct and indirect effects first by OLS (assuming sequential ignorability), and then via both variants of the sample selection models, to compare estimates.
Future work will consider estimates by using an alternative instrument for education, in the framework of \cite{frolich2017direct} to avoid the modelling assumptions inherent to sample selection models.
