\section{Causal Mediation (CM) in Applied Settings}
\label{sec:applied}
In this section, I further develop the issue of selection in CM estimates. First, I show the non-parametric bias terms from above can be written as omitted variables bias in a regression framework.
Second, I show how selection bias operates in an applied model for selection into a mediator based on costs and benefits.

\subsection{Regression Framework}
\label{sec:regression}
Inference for CM effects can be written in a regression framework, showing how correlation between the error term and the mediator persistently biases estimates.

Start by writing potential outcomes $Y_i(., .)$ as a sum of observed and unobserved factors, following the notation of \cite{heckman2005structural}.
For each $z',d' = 0,1$, put $\mu_{d'}(z'; \vec X) = \Egiven{Y_i(z', d')}{\vec X}$ and the corresponding error terms, $U_{d', i} = Y_i(z', d') - \mu_{d'}(z'; \vec X)$, so we have the following expressions:
\[ Y_i(Z_i, 0)
        = \mu_{0}(Z_i; \vec X_i) + U_{0,i}, \;\;
    Y_i(Z_i, 1)
        = \mu_{1}(Z_i; \vec X_i) + U_{1,i}. \]
In these terms, the ADE and AIE are represented as follows,
\begin{align*}
    \text{ADE}
    %= \E{Y_i(1, D_i(Z_i)) - Y_i(0, D_i(Z_i))}
    % &= \E{ \mu_{D_i}(1; \vec X_i) - \mu_{D_i}(0; \vec X_i)}, \\
    &= \E{ D_i \Big( \mu_{1}(1; \vec X_i) - \mu_{1}(0; \vec X_i)\Big)
        + (1 - D_i) \Big( \mu_{0}(1; \vec X_i) - \mu_{0}(0; \vec X_i)\Big)}, \\
    \text{AIE}
    %= \E{Y_i(Z_i, D_i(1)) - Y_i(Z_i, D_i(0))}
        &= \E{\Big( D_i(1) - D_i(0) \Big)
        \times \Big( \mu_1(Z_i; \vec X_i) - \mu_0(Z_i; \vec X_i) + U_{1,i} - U_{0,i}\Big) }.
\end{align*}
With this notation, observed data $Z_i, D_i, Y_i, \vec X_i$ have the following outcome equations --- which characterise direct effects, indirect effects, and selection bias.
\begin{align}
    \label{eqn:parametric-firststage}
    D_i &= \phi + \pi Z_i + \zeta(\vec X_i) + \eta_i  \\
    \label{eqn:parametric-secondstage}
    Y_i &= \alpha + \beta D_i + \gamma Z_i + \delta Z_i D_i
    + \varphi(\vec X_i)
    + \underbrace{\left(1 - D_i \right)U_{0,i} + D_i U_{1,i}}_{
        \text{Correlated error term.}}
\end{align}
First-stage \eqref{eqn:parametric-firststage} is identified, with $\phi + \zeta(\vec X_i)$ the intercept, and $\pi$ the first-stage compliance rate (which may depend on $\vec X_i$).
Second-stage \eqref{eqn:parametric-secondstage} has the following definitions, and is not identified thanks to omitted variables bias.\footnote{
    See \autoref{appendix:regression-model} for the derivation.
}
\begin{enumerate}[label=\textbf{(\alph*)}]
    \item $\alpha = \E{\mu_0(0; \vec X_i)}$ and $\varphi(\vec X_i) = \mu_0(0; \vec X_i) - \alpha$ are the intercept terms.
    \item $\beta = \mu_1(0; \vec X_i) - \mu_0(0; \vec X_i)$ is the AIE local to $Z_i = 0$.
    \item $\gamma = \mu_0(1; \vec X_i) - \mu_0(0; \vec X_i)$ is the ADE local to $D_i = 0$.
    \item $\delta = \mu_1(1; \vec X_i) - \mu_0(1; \vec X_i)- \left( \mu_1(0; \vec X_i) - \mu_0(0; \vec X_i) \right)$ is the average interaction effect.
    \item $\left( 1 - D_i \right) U_{0,i} + D_i U_{1,i}$ is the disruptive error term.
\end{enumerate}

The ADE and AIE are averages of these regression coefficients.
\begin{align*}
    \text{ADE}
    %= \E{Y_i(1, D_i(Z_i)) - Y_i(0, D_i(Z_i))}
        &= \E{\gamma + \delta D_i}, \\
    \text{AIE}
    %= \E{Y_i(Z_i, D_i(1)) - Y_i(Z_i, D_i(0))}
        &= \E{ \pi \left( \beta +  \delta Z_i + \tilde U_i \right)},
        \;\;\;\; \text{ with } \tilde U_i
            = \underbrace{\Egiven{ D_i U_{1,i} - \left(1 - D_i \right)U_{0,i}}{
                \vec X_i, D_i(1) = 1, D_i(0) = 0}}_{
                    \text{Unobserved complier differences.}}.
\end{align*}
The ADE is a simple sum of the coefficients, while the AIE includes a group differences term because it only refers to the mediator compliers.\footnote{
    \autoref{sec:selectionmodel} returns to the issue of accounting for mediator compliers in identifying the average indirect effect.
}

By construction, $\vec U_i \coloneqq \left(U_{0, i}, U_{1, i} \right)$ is an unobserved confounder.
The regression estimates of $\beta, \gamma, \delta$ in second-stage \eqref{eqn:parametric-secondstage} give unbiased estimates only if $D_i$ is also conditionally ignorable: $D_i \indep  \vec U_{i} $.
If not, then estimates of CM effects suffer from omitted variables bias from failing to adjust for the unobserved confounder, $\vec U_i$.

\subsection{Selection on Costs and Benefits}
CM is at risk of bias because $D_i \indep  \left(U_{0, i}, U_{1, i} \right)$ is unlikely to hold in applied settings.
A separate identification strategy could disrupt the selection-into-$D_i$ based on unobserved factors, and lend credibility to the mediator ignorability assumption.
Without it, bias will persist, given how we conventionally think of selection-into-treatment.

Consider a model where individual $i$ selects into a mediator based on costs and benefits (in terms of outcome $Y_i$), after $Z_i, \vec X_i$ have been assigned.
In a natural experiment setting, an external factor has disrupted individuals selecting $Z_i$ by choice (thus $Z_i$ is ignorable), but it has not disrupted the choice to take mediator (thus $D_i$ is not ignorable).
Write $C_i$ for individual $i$'s costs of taking mediator $D_i$, and $\indicator{.}$ for the indicator function.
The Roy model has $i$ taking the mediator if the benefits exceed the costs,
\begin{equation}
    \label{eqn:roy-model}
    D_i \left( z' \right) = \indicator{
    \underbrace{Y_i\left( z', 1 \right) - Y_i\left( z', 0 \right)}_{\text{Benefits}}
    \geq \underbrace{C_i}_{\text{Costs}}}, \;\;\; \text{for } z'=0,1.
\end{equation}
The Roy model provides an intuitive framework for analysing selection mechanisms because it captures the fundamental economic principle of decision-making based on costs and benefits in terms of the outcome under study \citep{roy1951some,heckman1990empirical}.
If the outcome $Y_i$ is a measure of income, and the mediator a choice of taking education, then it models an individual choice to attend more education in terms of gaining a higher income compared to the costs.\footnote{
    If the choice is made for a sum of outcomes, then a simple extension to a utility maximisation model maintains this same framework.
    See \cite{heckman1990empirical,eisenhauer2015generalized}.
}
This makes it particularly useful as a base case for CM, where selection into the mediator may be driven by private information (unobserved by the researcher).
% Additionally, the Roy model aligns well with many real-world settings, such as education or labour market participation, where decisions are based on a comparison of expected outcomes across alternatives. 
By using the Roy model as a benchmark, I explore the practical limits of the mediator ignorability assumption.

Decompose the costs into its mean and an error term, $C_i(Z_i) = \mu_{C}(Z_i; \vec X_i) + U_{C,i}$, to give a representation of Roy selection in terms of observed and unobserved factors,
\[ D_i(z') = \indicator{
    \mu_1(z'; \vec X_i) - \mu_0(z'; \vec X_i) - \mu_C(z'; \vec X_i)
    \geq U_{C,i} - \Big(U_{1,i} - U_{0,i}\Big) }
        , \;\;\; \text{for } z'=0,1. \]

If selection is Roy style, and the mediator is ignorable, then unobserved benefits play no part in selection.
The only driver in differences in selection are differences in costs (and not benefits).
If there are any unobserved benefits for selection into $D_i$ unobserved to the researcher, then sequential ignorability cannot hold.
\begin{definition}
    \label{def:roy-seq-ig}
    Suppose mediator selection follows a Roy model \eqref{eqn:roy-model}, and selection is not fully explained by costs and observed gains.
    Then sequential ignorability does not hold.
\end{definition}
If there are any unobserved sources of gains, then sequential ignorability does not hold.
This is an equivalence statement: selection based on costs and benefits is only consistent with mediator ignorability if the researcher observed every single source of mediator benefits.
See \autoref{appendix:roy-seq-ig} for the proof.

This means than the vector of control variables $\vec X_i$ must be incredibly rich.
Together, $\vec X_i$ and unobserved cost differences $U_{C,i}$ must explain selection into $D_i$ one hundred percent.
In the Roy model framework, however, individuals make decisions about mediator take-up based on gains, which the researcher may not observe fully. 
These unobservables are unlikely to be fully captured by an observed control set $\vec X_i$, except in very special cases (see e.g., the discussion in \citealt{angrist2009mostly,angrist2022empirical}).
%\footnote{
%    In a similar sense, \cite{huber2024testing} give a method to  test the implications of sequential ignorability (requiring an instrument).
%}
In practice, the only way to believe in the ignorability assumption is to study a setting where the researcher has a causal research design for both treatment $Z_i$ and mediator $D_i$, at the same time.
A simple addition of ``we assume the mediator satisfies selection-on-observables'' will not cut it here, and will lead to biased inference in practice.
%Consequently, the assumption of mediator ignorability is implausible in most practical settings.

% \subsection{Applied Settings}
% 
% Three parapgraphs on what goes on in empirical settings.
% Survey the papers, and speak about it heavily in one paragraph.
% 
% table:
% 
% name | $Z \to Y$ | design for $Z$ | Primary mediatory | controls | Possible $U$.
