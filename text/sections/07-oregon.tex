\section{CM in the Oregon Health Insurance Experiment}
\label{sec:oregon}
In the Oregon Health Insurance Experiment, winning the wait-list lottery significantly improved subjective health and well-being among participants. 
This study investigates the mechanisms behind these benefits, quantifying the extent to which improvements are mediated through increased healthcare usage.

To address concerns for unobserved selection-into-healthcare, I use the respondents' regular healthcare provider before the wait-list lottery as an IV for healthcare usage.
Approximately 73.2\% reported visiting a healthcare provider within the past year, but rates vary notably depending on their provider type: those who reported attended hospital emergency rooms (A\&E) and urgent care clinics before the wait-list lottery reported significantly lower healthcare visitation rates after the lottery (8.4 and 11 percentage points lower, respectively) than the 40\% who attended private clinics.\footnote{
    The combined $F$ statistic for the categorical variable for healthcare usual location (before the wait-list lottery) on healthcare usage (following the lottery) is 38.4.
}
The IV validity arises from differential costs faced by individuals based on their usual care provider.
Private clinics generally charge through health insurance and are more expensive without coverage, while A\&E and urgent care often provide costly services but rarely follow up on unpaid bills, effectively creating variation in healthcare attendance costs.
Additionally, individuals' choice of provider likely depend on neighbourhood-based access.

Initial results with unadjusted CM estimates suggest almost no mediating role for healthcare usage; the unadjusted estimates of the AIE are close to zero for both outcomes, contradicting intuitive suggestive evidence.
These estimates control for diagnosis of serious health conditions (before the wait-list lottery), such as kidney disease or diabetes.
However, this approach would not be considering possible unobserved confounding coming from underlying health conditions.
My MTE-based approach attempts to model this unobserved confounding, and applying them to these data reveals a much larger, positive AIE, restoring the mediating mechanism of healthcare usage in line with suggestive intuition.
This is because a correlational estimate of health and well-being gains to healthcare visits are practically zero, while the IV estimates restore positive gains and the MTE-based methods pick this up with a larger AIE estimate.
These numbers are reported in \autoref{tab:cm-oregon}, where panel A shows the CM effects with the binary outcome of subjective health, and panel B the binary outcome of subjective well-being.

\begin{table}[h!]
    \singlespacing
    \centering
    \small
    \caption{CM Effect Estimates for Wait-list Lottery Effects on Health and Well-being.}
    \begin{tabular}{l c c c c c}
        \\[-1.8ex]\hline \hline \\[-1.8ex] 
        & First-stage & ATE & ADE & AIE & AIE / ATE \\
        \cmidrule(lr){2-6}
        & (1) & (2) & (3) & (4) & (5) \\
        \midrule
        \multicolumn{1}{l}{\textbf{Panel A:} Health overall good?} \\
        % latex table generated in R 4.5.1 by xtable 1.8-4 package
% Tue Oct 21 18:02:15 2025
 Unadjusted &  4.700 &  4.500 &  4.700 & -0.190 & -0.043 \\ 
   & (0.880) & (0.910) & (0.910) & (0.063) & (0.018) \\ 
  Parametric MTE & 4.70 & 5.00 & 2.50 & 1.80 & 0.36 \\ 
   & (0.88) & (0.95) & (1.20) & (0.55) & (0.14) \\ 
  Semi-parametric MTE & 4.70 & 5.00 & 2.00 & 3.10 & 0.61 \\ 
   & (0.84) & (0.97) & (1.30) & (0.87) & (0.22) \\ 
  
        \\[-1.8ex]\hline \\[-1.8ex]
        \multicolumn{1}{l}{\textbf{Panel B:} Happy overall?} \\
        % latex table generated in R 4.5.1 by xtable 1.8-4 package
% Tue Oct 21 18:16:11 2025
 Unadjusted & 4.7000 & 7.1000 & 7.0000 & 0.0670 & 0.0094 \\ 
   & (0.8600) & (0.9600) & (0.9600) & (0.0520) & (0.0077) \\ 
  Parametric MTE & 4.70 & 7.50 & 5.00 & 1.90 & 0.25 \\ 
   & (0.870) & (0.990) & (1.100) & (0.510) & (0.075) \\ 
  Semi-parametric MTE & 4.70 & 7.50 & 5.00 & 2.50 & 0.34 \\ 
   & (0.84) & (0.93) & (1.20) & (0.74) & (0.11) \\ 
  
        \\[-1.8ex]\hline \\[-1.8ex]
    \end{tabular}
    \vspace{-0.125cm}
    \label{tab:cm-oregon}
    \justify
    \footnotesize
    \textbf{Note:}
    This table shows the point estimates (and SEs in brackets) of applying the proposed CM methods to replication data from the Oregon Health Insurance Experiment \citep{icspr2014oregon}.
    The first-stage column to the average effect of winning the wait-list lottery on healthcare usage (mediator first-stage), ATE average effect on surveyed health and well-being, ADE and AIE to respective CM effects through and absent healthcare usage.
    SEs were calculated with 1,000 bootstrap replications.
    The numbers are pp increases in the binary outcome, so an estimate of 4.1 in row 1 column 1 means an increase in 4.1 pp of using healthcare in the last 12 months after winning the wait-list lottery.
\end{table}

This reversal in conclusions highlights the importance of correcting for negative selection into healthcare usage.
A conventional approach to CM fails to account for the fact that individuals with poorer underlying health tend to visit healthcare providers more frequently, generating negative selection bias that obscures the true positive AIE, and clouds inference on the mediator mechanism.
By explicitly adjusting for this bias using the MTE approach, I isolate a credible positive indirect effect of healthcare usage on subjective health and well-being.

These findings offer credible evidence that improved healthcare access yields meaningful subjective health and well-being benefits, despite previous research emphasising negligible effects on objective health measures such as blood pressure \citep{baicker2013oregon}.
Subjective measures likely reflect broader psychological and financial relief associated with reduced healthcare-related anxiety and diminished risk of catastrophic medical debt, thus producing more noticeable short-term subjective improvements.

Nevertheless, this analysis is subject to notable limitations.
The IV is not ideal, and potentially more important mediators (such as explicit health insurance status) would require additional IVs beyond the wait-list lottery itself, presenting a challenging identification issue.
Furthermore, the 95\% confidence intervals for both ADE and AIE estimates (based on boostrapped SEs) remain large, though statistically significant and excluding zero.
This uncertainty underscores common challenges in applied CM analyses, where statistical precision can be limited by data constraints.
%my approach remains honest regarding the statistical uncertainty in gauging exactly how much the mediator mechanism channels explains the average causal effect. 
