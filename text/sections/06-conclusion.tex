%%%%%%%%%%%%%%%%%%%%%%%%%%%%%%%%%%%%%%%%%
%% Conclusion section
\section{Summary and Concluding Remarks}
\label{sec:conclusion}

This paper has studied a selection-on-observables approach to CM in a natural experiment setting.
I have shown the pitfalls of using the most popular methods for estimating direct and indirect effects without a clear case for the mediator being ignorable.
Using the Roy model as a benchmark, a mediator is unlikely to be ignorable in natural experiment settings, and the bias terms likely crowd out inference regarding CM effects.

This paper has contributed to the growing CM literature in economics, integrating labour economic theory for selection-into-treatment as a way of judging the credibility of conventional CM analyses.
It has drawn on the classic literature, and pointed to already-in-use control function methods as a compelling way of estimating direct and indirect effects in a natural experiment setting.
Further research could build on this approach by suggesting efficiency improvements, adjustments for common statistical irregularities (say, cluster dependence), or integrating the selection model/control function as an additional robustness in the growing double robustness literature \citep{farbmacher2022causal,bia2024double}.

This paper does not provide a blanket endorsement for applied researchers to use CM methods.
The structural assumptions are strong, and design-based inference requires an instrument for mediator take-up; if the assumptions are broken, then selection-adjusted estimates of CM effects will also be biased, and will not improve on the selection-on-observables approach.
And yet, there are likely settings in which the structural assumptions are credible.
Mediator monotonicity aligns well with economic theory in many cases, and it is plausible for researchers to study big data settings with external variation in mediator take-up costs.
In these cases, this paper opens the door to identifying mechanisms behind treatment effects in natural experiment settings.
% The approach could be used in AB tests, where a firm randomises a treatment and costs of a suspected mediator (if they do not want to also randomise a mediator fully).
