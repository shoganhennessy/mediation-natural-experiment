%%%%%%%%%%%%%%%%%%%%%%%%%%%%%%%%%%%%%%%%%
%% Appendix section
% Set-up the section.
%\newpage
\appendix
\setcounter{table}{0}
\renewcommand{\thetable}{A\arabic{table}}
\setcounter{figure}{0}
\renewcommand{\thefigure}{A\arabic{figure}}

% Start appendix
\section{Appendix}
\label{appendix}
This project used computational tools which are fully open-source.
%As such, all code and data involved in this project are available at this project's Github repository, available at \url{https://github.com/shoganhennessy/state-faculty-composition}.
%They may be used for replication, or as the basis for further work, as needed.
Any comments or suggestions may be sent to me at \href{mailto:seh325@cornell.edu}{\nolinkurl{seh325@cornell.edu}}, or raised as an issue on the Github project.

A number of statistical packages, for the R language \citep{R2023}, made the empirical analysis for this paper possible.
\begin{itemize}
    \item \textit{Tidyverse} \citep{tidyverse} collected tools for data analysis in the R language.
    \item \textit{DoubleML} \citep{DoubleML2020} implemented doubly robust methods used in the empirical analysis. 
    \item \textit{GRF} \citep{athey2019generalized,grf} compiled forest computational tools for the R language.
    \item \textit{Stargazer} \citep{stargazer} provided methods to efficiently convert empirical results into presentable output in \LaTeX.
\end{itemize}

\subsection{Identification in Causal Mediation}
\label{appendix:identification}
\citet[Theorem~1]{imai2010identification} states that the direct and indirect effects are identified under sequential ignorability, at each level of $Z_i = 0,1$.
For $z' = 0,1$: \\
\makebox[\textwidth]{\parbox{1.25\textwidth}{
\begin{align*}
    \E{ Y_i(1, D_i(z')) - Y_i(0, D_i(z'))}
    &= \int \int 
    \Big( \Egiven{ Y_i }{Z_i = 1, D_i, \vec X_i}
        - \Egiven{ Y_i }{Z_i = 0, D_i, \vec X_i} \Big)
            dF_{D_i \, | \, Z_i = z', \vec X_i} dF_{\vec X_i}, \\
    \E{ Y_i(z', D_i(1)) - Y_i(z', D_i(0))}
    &= \int \int \Egiven{ Y_i }{Z_i = z', D_i, \vec X_i}
    \Big( dF_{D_i \, | \, Z_i = 1, \vec X_i}
        - dF_{D_i \, | \, Z_i = 0, \vec X_i} \Big) dF_{\vec X_i}.
\end{align*}
}}
I focus on the averages, which are identified by consequence of the above.
\begin{align*}
    \E{ Y_i(1, D_i(Z_i)) - Y_i(0, D_i(Z_i))}
    &= \E[Z_i]{\Egiven{ Y_i(1, D_i(z')) - Y_i(0, D_i(z'))}{Z_i = z'}} \\
    \E{ Y_i(Z_i, D_i(1)) - Y_i(Z_i, D_i(0))}
    & = \E[Z_i]{\Egiven{ Y_i(z', D_i(1)) - Y_i(z', D_i(0))}{Z_i = z'}}
\end{align*}
My estimand for the average direct effect is a simple rearrangement of the above.
The estimand for the average indirect effect relies on a different sequence, relying on (1) sequential ignorability, (2) conditional monotonicity.
These give (1) identification of, and equivalence between, LADE conditional on $\vec X_i$ and ADE conditional on $\vec X_i$, (2) identification of the complier score.

\begin{align*}
    & \Egiven{ Y_i(Z_i, D_i(1)) - Y_i(Z_i, D_i(0))}{\vec X_i} \\
    & = \Probgiven{D_i(1) = 1, D_i(0) = 0}{\vec X_i}
        \Egiven{ Y_i(Z_i, 1) - Y_i(Z_i, 0)}{D_i(1) = 1, D_i(0) = 0, \vec X_i} \\
    & = \Probgiven{D_i(1) = 1, D_i(0) = 0}{\vec X_i}
        \Egiven{ Y_i(Z_i, 1) - Y_i(Z_i, 0)}{\vec X_i} \\
    & = \Big( \Egiven{D_i}{Z_i = 1, \vec X_i} - \Egiven{D_i}{Z_i = 0, \vec X_i}
        \Big) \; \Egiven{ Y_i(Z_i, 1) - Y_i(Z_i, 0)}{\vec X_i} \\
    & = \Big( \Egiven{D_i}{Z_i = 1, \vec X_i} - \Egiven{D_i}{Z_i = 0, \vec X_i}
        \Big)
        \Big( \Egiven{Y_i}{Z_i, D_i = 1, \vec X_i}
            - \Egiven{Y_i}{Z_i, D_i = 0, \vec X_i} \Big)
\end{align*}
Monotonicity is not technically required for the above.
Breaking monotonicity would not change the identification of any of the above; it would be the same except replacing the complier score with a complier or defier score, $\Probgiven{D_i(1) \neq D_i(0)}{\vec X_i} = \Egiven{D_i}{Z_i = 1, \vec X_i} - \Egiven{D_i}{Z_i = 0, \vec X_i}$.

%\subsection{Continuous Average Causal Responses}
%\label{appendix:continuous}
%Section here relating the approach to the average causal response function (see e.g., Angrist Imbens JASA 1996, Andrew Bacon for DiD 2023).

% \subsection{Previous Literature}
% \label{appendix:mediation-review}
% 
% Create a table in this section that surveys previous research which employs mediation methods while having a clear causal design for $Z_i$, but not $D_i$.
%
%\begin{tabular}{l l l l l}
%    Paper & Field & Research Design for $Z_i$ & Research Design for $D_i$ & Selection bias? \\ \hline
%    Paper name 1.    
%\end{tabular}

\subsection{Bias in Mediation Estimates}
\label{appendix:mediation-bias}
Suppose that $Z_i$ is ignorable conditional on $\vec X_i$, but $D_i$ is not.

\subsubsection{Bias in Direct Effect Estimates}
To show that the conventional approach to mediation gives an estimate for the ADE with selection and group difference-bias, start with the components of the conventional estimands.
This proof starts with the relevant expectations, conditional on a specific value of $\vec X_i$.
For each $d' =0, 1$.
\begin{align*}
    \Egiven{Y_i}{Z_i = 1, D_i = d', \vec X_i}
    =& \Egiven{Y_i(1, D_i(Z_i))}{D_i(1) = d', \vec X_i}, \\
    \Egiven{Y_i}{Z_i = 0, D_i = d', \vec X_i}
    =& \Egiven{Y_i(0, D_i(Z_i))}{D_i(0) = d', \vec X_i}
\end{align*}
And so
\begin{align*}
    &  \Egiven{Y_i}{Z_i = 1, D_i = d', \vec X_i}
    - \Egiven{Y_i}{Z_i = 0, D_i = d', \vec X_i} \\
    =& \Egiven{Y_i(1, D_i(Z_i))}{D_i(1) = d', \vec X_i}
    - \Egiven{Y_i(0, D_i(Z_i))}{D_i(0) = d', \vec X_i} \\
    =& \Egiven{Y_i(1, D_i(Z_i)) - Y_i(0, D_i(Z_i))}{D_i(1) = d', \vec X_i} \\
    &+ \Egiven{Y_i(0, D_i(Z_i))}{D_i(1) = d', \vec X_i}
        - \Egiven{Y_i(0, D_i(Z_i))}{D_i(0) = d', \vec X_i} 
\end{align*}
The final term is a sum of the ADE, conditional on $D_i(1) = d'$, and a selection bias term --- difference in baseline terms between the (partially overlapping) groups for whom $D_i(1) = d'$ and $D_i(0) = d'$.

To reach the final term, note the following.
\begin{align*}
    &\Egiven{Y_i(1, D_i(Z_i)) - Y_i(0, D_i(Z_i))}{\vec X_i} \\    
    =& \Egiven{Y_i(1, D_i(Z_i)) - Y_i(0, D_i(Z_i))}{D_i(1) = d', \vec X_i} \\
    &+ \Big(1 - \Probgiven{D_i(1) = d'}{\vec X_i}\Big)
    \left( \begin{aligned}
        &\Egiven{Y_i(1, D_i(Z_i)) - Y_i(0, D_i(Z_i))}{D_i(1) = d', \vec X_i} \\ 
        & - \Egiven{Y_i(1, D_i(Z_i)) - Y_i(0, D_i(Z_i))}{D_i(1) = 1 - d', \vec X_i}
    \end{aligned} \right) 
\end{align*}
The second term is a difference term between the average and the average for relevant complier groups.

Collect everything together, as follows.
\begin{align*}
    &  \Egiven{Y_i}{Z_i = 1, D_i = d', \vec X_i}
    - \Egiven{Y_i}{Z_i = 0, D_i = d', \vec X_i} \\
    =& \underbrace{
        \Egiven{Y_i(1, D_i(Z_i)) - Y_i(0, D_i(Z_i))}{\vec X_i}}_{
            \text{ADE, conditional on }\vec X_i} \\
    &+ \underbrace{
        \Egiven{Y_i(0, D_i(Z_i))}{D_i(1) = d', \vec X_i}
            - \Egiven{Y_i(0, D_i(Z_i))}{D_i(0) = d', \vec X_i}}_{
                \text{Selection bias}} \\
    &+ \underbrace{\Big(1 - \Probgiven{D_i(1) = d'}{\vec X_i}\Big)
    \left( \begin{aligned}
        &\Egiven{Y_i(1, D_i(Z_i)) - Y_i(0, D_i(Z_i))}{D_i(1) = d', \vec X_i} \\ 
        & - \Egiven{Y_i(1, D_i(Z_i)) - Y_i(0, D_i(Z_i))}{D_i(1) = 1 - d', \vec X_i}
    \end{aligned} \right)}_{
        \text{group difference-bias}}
\end{align*}
The proof is achieved by applying the expectation across $D_i = d'$, and $\vec X_i$.

\subsubsection{Bias in Indirect Effect Estimates}
To show that the conventional approach to mediation gives an estimate for the AIE with selection and group difference-bias, start with the definition of the ADE --- the direct effect among compliers times the size of the complier group.

This proof starts with the relevant expectations, conditional on a specific value of $\vec X_i$.
\begin{align*}
    &\Egiven{ Y_i(Z_i, D_i(1)) - Y_i(Z_i, D_i(0))}{\vec X_i} \\
    =& \Probgiven{D_i(1) = 1, D_i(0) = 0}{\vec X_i}
        \Egiven{ Y_i(Z_i, 1) - Y_i(Z_i, 0)}{D_i(1) = 1, D_i(0) = 0, \vec X_i}
\end{align*}
When $D_i$ is not ignorable, the bias comes from estimating the second term,\\ $\Egiven{ Y_i(Z_i, 1) - Y_i(Z_i, 0)}{D_i(1) = 1, D_i(0) = 0, \vec X_i}$.

For each $z' =0, 1$.
\begin{align*}
    \Egiven{Y_i}{Z_i = z', D_i = 1, \vec X_i}
    =& \Egiven{Y_i(z', 1)}{D_i = 1, \vec X_i}, \\
    \Egiven{Y_i}{Z_i = z', D_i = 0, \vec X_i}
    =& \Egiven{Y_i(z', 0)}{D_i = 0, \vec X_i}
\end{align*}
So compose the CM estimand, as follows.
\begin{align*}
    & \Egiven{Y_i}{Z_i = z', D_i = 1, \vec X_i}
    - \Egiven{Y_i}{Z_i = z', D_i = 0, \vec X_i} \\
    =& \Egiven{Y_i(z', 1)}{D_i = 1, \vec X_i}
        - \Egiven{Y_i(z', 0)}{D_i = 0, \vec X_i} \\
    =& \Egiven{Y_i(z', 1) - Y_i(z', 0)}{D_i = 1, \vec X_i}
    + \Egiven{Y_i(z', 0)}{D_i = 1, \vec X_i} - \Egiven{Y_i(z', 0)}{D_i = 0, \vec X_i}
\end{align*}
The final term is a sum of the AIE, among the treated group $D_i = 1$, and a selection bias term --- difference in baseline terms between the groups $D_i = 1$ and $D_i = 0$.

The AIE is the direct effect among compliers times the size of the complier group, so we need to compensate for the difference between the treated group $D_i = 1$ and complier group $D_i(1)= 1, D_i(0) = 0$.

Start with the difference between treated group's average and overall average.
\begin{align*}
    & \Egiven{Y_i(z', 1) - Y_i(z', 0)}{D_i = 1, \vec X_i} \\
    =& \Egiven{Y_i(z', 1) - Y_i(z', 0)}{\vec X_i} \\
    &+ \Big(1 - \Probgiven{D_i = 1}{\vec X_i} \Big)
    \left( \begin{aligned}
        &\Egiven{Y_i(z', 1) - Y_i(z', 0)}{D_i = 1, \vec X_i} \\ 
        &  - \Egiven{Y_i(z', 1) - Y_i(z', 0)}{D_i = 0, \vec X_i}
    \end{aligned} \right)
\end{align*}
Then the difference between the compliers' average and the overall average.
\begin{align*}
    & \Egiven{ Y_i(z', 1) - Y_i(z', 0)}{D_i(1) = 1, D_i(0) = 0, \vec X_i} \\
    =& \Egiven{Y_i(z', 1) - Y_i(z', 0)}{\vec X_i} \\
    & + \frac{1 - \Probgiven{D_i(1) = 1, D_i(0) = 0}{\vec X_i} }{
        \Probgiven{D_i(1) = 1, D_i(0) = 0}{\vec X_i}}
    \left( \begin{aligned}
        &\Egiven{Y_i(z', 1) - Y_i(z', 0)}{D_i(1) = 0 \text{ or } D_i(0)=1, \vec X_i} \\ 
        &  - \Egiven{Y_i(z', 1) - Y_i(z', 0)}{\vec X_i}
    \end{aligned} \right)
\end{align*}

Collect everything together, as follows.
\begin{align*}
    &  \Egiven{Y_i}{Z_i = z', D_i = 1, \vec X_i}
    - \Egiven{Y_i}{Z_i = z', D_i = 0, \vec X_i} \\
    =& \underbrace{
        \Egiven{Y_i(z', D_i(1)) - Y_i(z', D_i(0))}{\vec X_i}}_{
            \text{AIE, conditional on }\vec X_i, Z_i = z'} \\
    &+ \underbrace{
        \Egiven{Y_i(z', 0)}{D_i = 1, \vec X_i}
            - \Egiven{Y_i(z', 0)}{D_i = 0, \vec X_i}}_{
                \text{Selection bias}} \\
    &+ \underbrace{\left[ \begin{aligned}
        &\Big(1 - \Probgiven{D_i = 1}{\vec X_i} \Big)
        \left( \begin{aligned}
            &\Egiven{Y_i(z', 1) - Y_i(z', 0)}{D_i = 1, \vec X_i} \\ 
            &  - \Egiven{Y_i(z', 1) - Y_i(z', 0)}{D_i = 0, \vec X_i}
        \end{aligned} \right) \\
        &+ \frac{1 - \Probgiven{D_i(1) = 1, D_i(0) = 0}{\vec X_i} }{
            \Probgiven{D_i(1) = 1, D_i(0) = 0}{\vec X_i}} 
        \left( \begin{aligned}
            &\Egiven{Y_i(z', 1) - Y_i(z', 0)}{D_i(1) = 0 \text{ or } D_i(0)=1, \vec X_i} \\ 
            &  - \Egiven{Y_i(z', 1) - Y_i(z', 0)}{\vec X_i}
        \end{aligned} \right)
    \end{aligned} \right] }_{
        \text{group difference-bias}}
\end{align*}
The proof is finally achieved by multiplying by the complier score, 
$\Probgiven{D_i(1) = 1, D_i(0) = 0}{\vec X_i}$
$= \Egiven{D_i}{Z_i = 1, \vec X_i} - \Egiven{D_i}{Z_i = 0, \vec X_i}$,
then applying the expectation across $Z_i = z'$, and $\vec X_i$.

\subsection{A Regression Framework for Direct and Indirect Effects}
\label{appendix:regression-model}
Put $\mu_{D}(Z; \vec X) = \Egiven{Y_i(Z, D)}{\vec X}$ and $U_{D, i} = Y_i(Z,D) - \mu_D(Z; \vec X)$, so we have the following expressions.
\[ Y_i(Z_i, 0)
        = \mu_{0}(Z_i; \vec X_i) + U_{0,i}, \;\;
    Y_i(Z_i, 1)
        = \mu_{1}(Z_i; \vec X_i) + U_{1,i} \]

$U_{0,i}, U_{1,i}$ are error terms with unknown distributions, mean independent of $Z_i, \vec X_i$ by definition --- but possibly correlated with $D_i$.

$Z_i$ is independent of potential outcomes, so that $U_{0,i}, U_{1,i} \indep Z_i$.
Thus, the first-stage regression of $Z \to Y$ has unbiased estimates.
\begin{align*}
    D_i &= Z_i D_i(1) + (1 - Z_i) D_i(0) \\
        &= D_i(0) +
            Z_i \left[ D_i(1) - D_i(0) \right] \\
        &= \underbrace{\Egiven{D_i(0)}{\vec X_i}        
        }_{\text{Intercept}} +
            \underbrace{Z_i \E{ D_i(1) - D_i(0)}}_{
                \text{Regressor}} \\
            & \;\;\;\; + \underbrace{
                D_i(0) - \Egiven{D_i(0)}{\vec X_i}
                + Z_i \big( D_i(1) - D_i(0) - \Egiven{ D_i(1) - D_i(0)}{\vec X_i}\big)}_{
                \text{Mean-zero independent error term, since }Z_i \indep D_i \; | \; \vec X_i} \\
        &\eqqcolon \phi + \pi Z_i + \varphi(\vec X_i) + \eta_i \\
    \implies \Egiven{D_i}{Z_i, \vec X_i}&=
        \phi + \pi Z_i + \varphi(\vec X_i)
        \text{, and thus unbiased estimates since } Z_i \indep \phi, \eta_i.
\end{align*}

$Z_i$ is also assumed independent of potential outcomes $Y_i(,.,)$, so that $U_{0,i}, U_{1,i} \indep Z_i$.
Thus, the reduced form regression $Z \to Y$ also leads to unbiased estimates.

The same cannot be said of the regression that estimates direct and indirect effects, without further assumptions.
\begin{align*}
    Y_i &= Z_i Y_i(1, D_i(1)) + (1 - Z_i) Y_i(0, D_i(0)) \\
        &= Z_i D_i Y_i(1, 1) \\
        & \;\;\;\; + (1 - Z_i) D_i Y_i(0, 1) \\
        & \;\;\;\; + Z_i (1 - D_i) Y_i(1, 0) \\
        & \;\;\;\; + (1 - Z_i) (1 - D_i) Y_i(0, 0) \\
        &= Y_i(0, 0) \\
        & \;\;\;\; + Z_i \left[Y_i(1, 0) - Y_i(0, 0) \right] \\
        & \;\;\;\; + D_i \left[Y_i(0, 1) - Y_i(0, 0) \right] \\
        & \;\;\;\; + Z_i D_i \left[Y_i(1, 1) - Y_i(1, 0)
            - \left( Y_i(0, 1) - Y_i(0, 0) \right)\right]
\end{align*}
And so $Y_i$ can be written as a regression equation in terms of the observed factors and error terms.
\begin{align*}
    Y_i &= \mu_0(0; \vec X_i) \\
        & \;\;\;\; + D_i \left[\mu_1(0; \vec X_i) - \mu_0(0; \vec X_i) \right] \\
        & \;\;\;\; + Z_i \left[\mu_0(1; \vec X_i) - \mu_0(0; \vec X_i) \right] \\
        & \;\;\;\; + Z_i D_i \left[\mu_1(1; \vec X_i) - \mu_0(1; \vec X_i)
            - \left( \mu_1(0; \vec X_i) - \mu_0(0; \vec X_i) \right)\right] \\
        & \;\;\;\; + U_{0,i} + D_i \left( U_{1,i} - U_{0,i} \right) \\
        &\eqqcolon
            \alpha + \beta D_i + \gamma Z_i + \delta Z_i D_i
            + \zeta(\vec X_i)
            + \left( 1 - D_i \right) U_{0,i} + D_i U_{1,i}
\end{align*}
With the following definitions:
\begin{enumerate}[label=\textbf{(\alph*)}]
    \item $\alpha = \E{\mu_0(0; \vec X_i)}$ and $\zeta(\vec X_i) = \mu_0(0; \vec X_i) - \alpha$ are the intercept terms.
    \item $\beta = \mu_1(0; \vec X_i) - \mu_0(0; \vec X_i)$ is the indirect effect under $Z_i = 0$
    \item $\gamma = \mu_0(1; \vec X_i) - \mu_0(0; \vec X_i)$ is the direct effect under $D_i = 0$.
    \item $\delta = \mu_1(1; \vec X_i) - \mu_0(1; \vec X_i)- \left( \mu_1(0; \vec X_i) - \mu_0(0; \vec X_i) \right)$ is the interaction effect.
    \item $\left( 1 - D_i \right) U_{0,i} + D_i U_{1,i}$ is the remaining error term.
\end{enumerate}
This sequence gives us the resulting regression equation:
\begin{align*}
    \Egiven{Y_i}{Z_i, D_i, \vec X_i} =&
        \alpha
        + \beta D_i
        + \gamma Z_i
        + \delta Z_i D_i
        + \zeta(\vec X_i) \\
        & +\left( 1 - D_i \right) \Egiven{ U_{0,i} }{D_i = 0, \vec X_i}
            + D_i \Egiven{ U_{1,i} }{D_i = 1, \vec X_i}
\end{align*}
Taking the conditional expectation, and collecting for the expressions of the direct and indirect effects:\footnote{
    These equations have simpler expressions after assuming constant treatment effects in a linear framework;
    I have avoided this as having compliers, and controlling for observed factors $\vec X_i$ only makes sense in the case of heterogeneous treatment effects.
}
\begin{align*}
    \E{Y_i(Z_i, D_i(1)) - Y_i(Z_i, D_i(0))}
        &= \E{\pi \left( \beta +  Z_i \delta \right)} \\
    \E{Y_i(1, D_i(Z_i)) - Y_i(0, D_i(Z_i))}
        &= \E{\gamma + \delta D_i}
\end{align*}
These terms are conventionally estimated in a simultaneous regression \citep{imai2010identification}.

If sequential ignorability does not hold, then the regression estimates from estimating the mediation equations (without adjusting for the contaminated bias term) suffer from omitted variables bias.

\makebox[\textwidth]{\parbox{1.25\textwidth}{
\begin{align*}
    \E[\vec X_i]{\Egiven{Y_i}{Z_i = D_i = 0, \vec X_i}}
        &= \E\alpha + \Egiven{ U_{0,i} }{D_i = 0} \\
    \E[\vec X_i]{\Egiven{Y_i}{Z_i = 0, D_i = 1, \vec X_i}
        - \Egiven{Y_i}{Z_i = 0, D_i = 0, \vec X_i}}
        &= \E\beta + 
            \left( \Egiven{ U_{1,i} }{D_i = 1} - \Egiven{ U_{0,i} }{D_i = 0} \right) \\
    \E[\vec X_i]{\Egiven{Y_i}{Z_i = 1, D_i = 0, \vec X_i}
        - \Egiven{Y_i}{Z_i = 0, D_i = 0, \vec X_i}}
        &= \E\gamma + \Egiven{ U_{0,i} }{D_i = 0} \\
    \E[\vec X_i]{\begin{aligned}
        &\Egiven{Y_i}{Z_i = 1, D_i = 1, \vec X_i}
            - \Egiven{Y_i}{Z_i = 1, D_i = 0, \vec X_i} \\
            &- \left( \Egiven{Y_i}{Z_i = 0, D_i = 1, \vec X_i}
                - \Egiven{Y_i}{Z_i = 0, D_i = 0, \vec X_i} \right)
        \end{aligned}}
    &= \E\delta
\end{align*}
}}
And so the direct and indirect effect estimates are contaminated by these bias terms.

% Senan note: should write CM estimand = \E{\gamma + \delta D_i} + bias, where bias is some E[ U | D = 1] term.



\subsection{Control Function Identification}
\label{appendix:controlfun-proof}
Write the proof in here, following \cite{vytlacil2002independence} construction in the forward direction.
Note that the notation needs updating for no exclusion restriction.

And then 
